\documentstyle{book}
\begin{document}
%%%%%%%%%%%%%%%%%% title page %%%%%%%%%%%%%%%%%%%%%

\title{NICOLE \\
Non-LTE Inversion COde using \\
the Lorien Engine}

\author{Hector Socas-Navarro,
    Jaime de la Cruz \& Andr\'es Asensio Ramos \\
%High Altitude Observatory\\
%3080 Center Green Dr (CG-1)\\
%Boulder, CO (USA) 80301\\
%e-mail: \mbox{navarro@ucar.edu}
%}
%
%\author{Andr\'es Asensio Ramos\\
%Instituto de Astrof\' \i sica de Canarias\\
%Avda V\' \i a L\' actea S/N\\
%La Laguna 38200, Tenerife (SPAIN)\\
e-mail: \mbox{hsocas@iac.es; jaime@astro.su.se; aasensio@iac.es}\\
}

\date{Version 14.12}

\maketitle

%%%%%%%%%%%%%%%%%%%%%%%%%%%%%%%%%%%%%%%%%%%%%%%%%%%%

\tableofcontents


%%%%%%%%%%%%%%%%%%%%%%%%%%%%%%%%%%%%%%%%%%

\chapter{Introduction}
\section{What is NICOLE?}

NICOLE is a general-purpose synthesis and inversion code for the
Stokes profiles emergent from solar/stellar atmospheres.  Solar
instrumentation is becoming more sophisticated every day and the data
sets which are currently available (as well as those expected for the
near future) demand the use of modern diagnostic tools in order to
retrieve as much information as possible from the observations.  The
algorithm is described in Socas-Navarro et al (2012, in preparation;
see also Socas-Navarro, Trujillo Bueno and Ruiz Cobo 2000). It seeks
for the model atmosphere that provides the best fit to the profiles
(in a least-squares sense) of an arbitrary number of
simultaneously-observed spectral lines.  The underlying hypotheses
are:

\begin{itemize}
\item Atomic level populations in statistical equilibrium (NLTE),
  assuming complete angle and frequency redistribution.
\item No sub-pixel atmospheric structure is considered, except for a
  filling factor and a prescribed external atmosphere.
\item The observed Stokes profiles are induced by the Zeeman 
       effect in transitions where L-S
        coupling is a valid approximation (with the exception of the
        infrared FeI line at 1565.28nm, which has an ad-hoc treatment).
\item The hydrostatic equilibrium equation is used to calculate the
  gas density and the height scale in the atmosphere. This is optional
  in synthesis mode (in that case, the gas pressure/density would be
  read from the input model) and mandatory in inversion mode
\item Undocumented feature for hyperfine structure calculations
  (please, check with the author for information)
\end{itemize}

The inversion core used for the development of NICOLE is the 
LORIEN engine (the Lovely
Reusable Inversion ENgine, also publicly available for 
download from the C.I.C. web page),
which combines the SVD technique with the Levenberg-Marquardt 
minimization method to
solve the inverse problem (see Press {\em et al.} 1990).

I would like to hear comments about people using NICOLE, where you
are, what your research is about and what your overall experience with
the code is. Please, drop me a line at hsocas@iac.es. If you have
complains, criticism or generally speaking have negative things to
say, please include the word ``cialis'' in your message subject ;)

%%%%%%%%%%%%%%%%%%%%%%%%%%%%%%%%%%%%%%%%%%%%%%%%%%%%%%%%%%%%%%%%%

\section{Requirements}
\begin{enumerate}
\item A Fortran 90 compiler, or above (e.g., F95, 2000,
  2003...). Strictly speaking, NICOLE uses F2003 but it uses only a
  very small subset of the 2003 standard which is supported by most
  F90 compilers. So your F90 compiler should work.
\item Python, recommended version 2.6.4 or above (although limited
  testing has shown that it works on 2.4 as well)
\item Some routines (see section~\ref{sec:compiling}) from the
  Numerical Recipes (Press {\em et al.} 
  1990.) book. You can type them directly from the book, obtain the
  distribution on a CD-ROM or download them from their website
  http://www.nr.com. 
\end{enumerate}

\section{Features}
Features of this code:
\begin{enumerate}
\item Written entirely in Fortran 90, taking advantage of
  the advanced capabilities provided by this programming language. The
  code is strictly compliant with the Fortran 2003 ANSI standard. It
  makes use of a very widely implemented extension to F90, namely
  allocatable arrays in derived data types. While this is not strictly
  standard in F90, most compilers are supporting it anyway. What all
  of this means in practice is that there is a slim chance that a
  particular F90 or F95 compiler will not compile NICOLE, but this
  would be a rare occurrence. Any compiler that is compatible with
  Fortran 2003 (or higher) should be able to compile NICOLE without a
  problem since the code is strictly F03 compliant.
\item Dynamic memory management. NICOLE makes use of the heap storage
  capabilities of Fortran 90 to dynamically allocate and deallocate
  memory during the program execution. This means that you no longer
  have to worry about the array dimensions, as in old F77 code, where
  you had to recompile your code every time you changed the
  dimensionality of the problem.  Instead, NICOLE will allocate and
  use the required memory during run time.
\item Modular, top-to-bottom design. There are no common blocks in
  NICOLE (almost) and the source code is clear and straightforward to
  understand and modify. Most of the program building blocks are
  encapsulated in modules so it is possible to take bits and pieces of
  NICOLE to use in your own code.
\item Easy to use. You don't even need to compile it.  Pre-compiled
  executable files will be distributed for the most popular hardware
  platforms.  Just download it and run it on your data.
\item Cross-platform. A Python script is included that analyzes the
  system it is running on and produces a suitable makefile for that
  platform (see compiling information below). Byte-endianness is taken
  into account to allow the code to use files written both in
  big-endian and little-endian machines (the code will always write
  little-endian files, regardless of the platform) in a way that is
  completely transparent to the user.
\item Distributed under the GPL license version 3 (except for the
  Numerical Recipes routines). The full license is at:
\begin{verbatim}
http://www.gnu.org/copyleft/gpl.html
\end{verbatim}
Basically this means that NICOLE is free and open source. You can
copy, edit, share or distribute it. We only ask that you give the
authors proper credit. There is a paper currently in
preparation. Please, cite us!
\end{enumerate}

\section{Credits}
\label{sec:credits}

This code borrows significantly from previous efforts. Most of the
actual code has been redesigned and rewritten, but not all. The
inversion part of NICOLE is based on the concept of the SIR code,
written by Dr. B. Ruiz Cobo (Ruiz Cobo and del Toro Iniesta 1992).

The synthesis part involves the computation of NLTE populations. The
package of routines that does this (in the forward/NLTE/ directory)
employs a design that is similar to that of Carlsson's MULTI v2.2
(Scharmer \& Carlsson 1985). By this I mean that the organization of
loops and the order in which physical quantities are calculated and
used is the same as in MULTI. Studying MULTI has been a great help
since figuring out an efficient design is what took most of the actual
work for this module. Most variable names have been maintained to
facilitate code readability for users who are familiar with
MULTI. Some routines (listed below) have been adapted directly from
MULTI for compatibility with its model atom files. Examples are the
routines for calculation of collisional rates. However, the core of
the iterative algorithm is different from that implemented in
MULTI. Instead of using the linearization scheme, NICOLE implements
preconditioning with a local operator (Rybicki \& Hummer 1991), as
discussed in Socas-Navarro \& Trujillo Bueno (1998). The solution of
the radiative transfer equation is also different. NICOLE uses the
short characteristics method.

Other routines that have been contributed to the project by various
scientists around the world are the following:
\begin{itemize}
  \item A few routines have been taken from the Numerical Recipes book
    (Press {\em et al.} 1986). These are detailed in
    section~\ref{sec:compiling} below. Note that I am not allowed to
    include the source code of the Numerical Recipes routines in the
    NICOLE distribution.
  \item The Zeeman splitting is computed by an old routine that I
    inherited at some point. I'm not entirely sure of this but I
    believe that this routine was originally written by
    A. Wittmann. If someone could confirm this, I'd appreciate it.
  \item The following routines have been adapted from M. Carlsson's
    MULTI: CA2COL, GENCOL, ENEQ \& NGFUNC (and their dependencies).
\end{itemize}


Disclaimer: This software is distributed ``as is'' and the authors
take no responsibility for any consequence derived from its use.

%%%%%%%%%%%%%%%%%%%%%%%%%%%%%%%%%%%%%%%%%%%%%%%%%%%%%%%%%%%%%%%%%%%%%%%%%5

\chapter{Note for users of previous versions}

\section{For users of versions prior to 2.6}

If you have never used NICOLE before or if you plan to read this
manual before working with the code, feel free to skip this chapter
altogether.  There are two important changes over previous
versions. Starting with this version, the native (binary) model file
format has changed to accomodate chemical abundances as part of the
model. This has been implemented to allow for the new feature of
abundance inversions. This means that file formats prior to 2.6 will
be converted by the Python wrapper (run\_nicole.py) into the new
format, using the abundances specified in the NICOLE.input files (or
selecting the default option of Grevesse \& Sauval 1998). The other
big change has to do with the code structure. The loop in inversion
cycles has been moved to the outmost level of the main program and
redefined to something much more general. Previously, inversion cycles
were successive inversion runs varying the number of nodes. Now, the
user can change not only the nodes but virtually every other parameter
from one cycle to the next. This is done by defining several
NICOLE.input files, each one with a suffix corresponding to a
different cycle (e.g., NICOLE.input\_1, NICOLE.input\_2, ... etc). It
is then possible to do things like keeping the intermediate inversion
models resulting from each cycle, or using the synthetic profiles from
a cycle as an input to the next, etc. This flexibility allows the user
to do in one run of the Fortran code (thus needing only one job
submission in a supercomputer) tasks that otherwise would require
several separate runs. The nodes.dat file has been removed. The number
of nodes is now specified in each one of the NICOLE.input file in a
new section called (of course) [nodes].

Additionally, there are some other smaller changes as well. It is now
possible to introduce in the input model certain variables such as
electron pressure, gas pressure, density, Hydrogen number density
(nH), other forms of Hydrogen (nHminus, nHplus, nH2, nH2plus) and
tell the code to observe those values (the default behavior is to solve
the ionization equilibrium and the chemical equilibrium and compute
the variables from one of them, typically electron pressure,
overwritting all other variables). Obviously this only makes sense in
the synthesis or conversion modes but not in the inversion mode in
which the model atmosphere is modified in succesive iterations.

\section{For users of versions prior to 2.0}

If you have never used NICOLE before or if you plan to read this manual 
before working with the code, feel free to skip this chapter altogether.
Two things work very differently in the current version of NICOLE with
respect to those prior to 2.0. The first one has to do with the code
compilation. In order to maximize portability and also
to make life easier for the casual user, an automated tool written in
Python has been included that works similarly to the popular autoconf
tool in Linux. The user simply needs to run this Python program
(create\_makefile.py) and it will automatically scan the system for
compilers, look for the suitable options (record length and
byte-endianness for binary files) and even make some slight
changes to the code to ensure compatibility of data written in
multiple platforms. It will also produce either the serial version
(default) or the MPI parallel version (when run with the --mpi
flag). The tool recognizes the most popular F90 compilers and sets the
appropriate flags. If none of them is found in the system, the user
will have to specify manually the compiler and its options. Currently
supported compilers include those from GNU, Intel, IBM or the Portland
Group. See section~\ref{sec:compiling} for information on compiling
this version. A battery of tests has also been included to verify that
the code works properly (section~\ref{sec:compiling}).

The other important change has to do with the handling of input/output
files. The whole process has been restructured and simplified. The
actual Fortran code now works only with a very specific fixed binary
file format, both for inputs and outputs. A Python program
(run\_nicole.py) now does all the parsing of human readable files and
creates the fixed-format file that the Fortran code will use. The
input files with model atmospheres or observations can now be given in
many different and convenient formats, which will then be transformed
by the Python wrapper into NICOLE's own native binary format. This
means that the user can now supply data in formats that include ASCII
(same as in previous versions), IDL savefile or NICOLE's native
format. The plan is to include in the near future also support for
FITS files (e.g., to invert directly data from Hinode) or some
numerical simulation codes. The input files containing the basic run
parameters, spectral grid to use or spectral line data, abundances,
etc have now been merged into one single file using a popular parsing
standard. The format of this input file is now much more flexible and
the user can easily track all of the run parameters. The price to pay
for this increased flexibility and convenience is that NICOLE must now
be run through the Python wrapper. The user is NOT supposed to run the
Fortran executable directly. See section~\ref{sec:start} for more information.

The following input files have become obsolete and have been removed
(most of their functionality has been incorporated in NICOLE.input):
the wavelength grid, ABUND and NLTE\_lines. There is now no difference
between inversion and multiple inversion modes (and similarly between
syntehsis and multiple synthesis). A single inversion or synthesis is
simply one where npix=1.

A minor change is that the magnetic field is now given in terms of its
components instead of its modulus and angles as in previous
versions. This makes the inversion better conditioned. 

%%%%%%%%%%%%%%%%%%%%%%%%%%%%%%%%%%%%%%%%%%%%%%%%%%%%%%%%%%%%%%%%%%%%%%%%%5

\chapter{Quick start}
\label{sec:start}

This chapter is for the impatient type. The first step is obviously to
uncompress and unpack the distribution. Under UNIX, do:

\begin{verbatim}
tar zxf nicole*.tar.gz
\end{verbatim}

then copy the relevant Numerical Recipes routines to the
numerical\_recipes/ directory. The files to be copied are: convlv.f90,
four1.f90, fourrow.f90, nr.f90, nrtype.f90, nrutil.f90, pythag.f90,
realft.f90, ludcmp.f90, lubksb.f90, svbksb.f90, svdcmp.f90, tqli.f90
and twofft.f90. Once you have done this, go to the main/ directory to create 
the makefile and compile with 
\begin{verbatim}
./create_makefile.py
make clean
make nicole
\end{verbatim}
If you get any errors after running create\_makefile, refer to
chapter~\ref{sec:compiling} for instructions on how to manually
configure it. If you wish to include compiler options, e.g. to specify
optimization parameters, you may do so with the --otherflags
switch. For instance:
\begin{verbatim}
./create_makefile.py --otherflags='-fast -O3'
\end{verbatim}
If the code compiled successfully you may run the
included tests to make sure it works properly:
\begin{verbatim}
cd ../test
./run_tests.py
\end{verbatim}
To run the tests in non-interactive mode, e.g. on platforms where jobs
need to be submitted to a queue system (typically supercomputers),
refer to the procedure described in section~\ref{sec:compiling} below.

Note: Often times you may want to create a new makefile using the same
options as the last time. This is particularly useful e.g. when you
update the source code distribution with a new release of the code. If
the source code structure hasn't changed then you don't need to run
creat\_makefile.py at all. However, there are times when a new version
of NICOLE includes new source files or perhaps some files no longer
exist or have been moved. Thus, it is always recommended to run 
create\_makefile.py again every time you install a new version of NICOLE.
To save you some hassle, you can use the --keepflags option to instruct
create\_makefile.py to use exactly the same command-line arguments that
you used the last time. These arguments are saved as comments in the
makefile so they can be read and reused by create\_makefile.py at a later
time.

If everything went smoothly, go into the run/ directory.

Note: NICOLE works internally with binary files (the format is
described below). Binary files can be written in two different ways,
usually called big-endian and little-endian style. Some hardware
platforms, such as PCs (generally speaking, machines based on the
Intel processor architecture), use the little-endian format whereas
others (e.g., Motorola or PowerPC architectures) use big-endian. In
order to ensure compatibility among files created in different
machines, NICOLE will always read and write files consistently using
the little-endian form even when it runs on big-endian machines. This
process should be completely transparent to the user except that some
file transfer clients (typically some ftp programs) are too smart and
will modify binary files (swap bytes) when transferring between
machines with different endianness in an attempt to make the file
compatible with the target machine. Such modification is not necessary
for NICOLE files and in fact would result in file corruption. If you
experience trouble running the code after transferring model or
profile files via ftp, this is probably the reason.

\section{The synthesis mode}
\label{synthmode}

In this mode you will be computing synthetic spectra from one or more
prescribed model atmospheres. You will need the following files:

\subsection{The ATOM file}

This file is necessary only when we are computing NLTE lines. If you
are working in LTE only, you don't need to read this. The ATOM file
resides in the directory where NICOLE is executed. It has the same
format as the model atoms used in MULTI, {\bf except} that comments
are signaled by an exclamation mark (!)  instead of an asterisk
(*). If the ionization stage is in roman numbers, you might have to
change that too (i.e., replace ``Ca~II'' with ``Ca~2''). So if you got
your ATOM file from MULTI, make sure to edit it and replace the
asterisks with exclamation marks. Some model atoms have a GENCOL
section at the end, which supplies the parameters for the collisional
routine GENCOL. That section contains a grid of temperatures under the
key TEMP, which specifies the number of temperatures in the grid
(ntemp), and then the temperature grid itself, T(1:ntemp). Unlike
MULTI, NICOLE requires that ntemp and the temperature grid T(1:ntemp)
be all in one line, without any line breaks. Keep in mind that each
line in the model atom is limited to a maximum of 500 characters (this
limit can be changed in the declaration of array ColStr in
forward/NLTE.f90). Similarly, the rest of the temperature dependent
data in the GENCOL section must be in one line.

There are some MULTI features that are not supported in the current
NICOLE version. For example, you cannot have bound-bound transitions
with IW=1. ITRAD=4 is not supported either. Everything else should
work as in MULTI, although we have only tested the code with a 6-level
Ca and a 14-level O atom. If you find it to run (or not) with other
atoms, please let us know.

Only one ATOM file may be used in a given run. This means that
you are limited to the calculation of NLTE lines (as many as desired)
existing in the model atom employed plus an arbitrary number of lines
from any other element(s) treated in LTE. 

Blends are treated consistently by NICOLE and computing blended lines
does not require any additional configuration by the user. However,
the departure coefficients for the NLTE lines are computed neglecting
blends (usually a good approximation unless such blends produce
significant distortions of a NLTE line core).

Note: The calculation of the magnetic field inclination and azimuth in
the line-of-sight coordinates has a singularity when both components
are zero. To avoid numerical accuracy issues, a magnetic field
component is considered zero by the code when it is below a threshold
of 10$^{-6}$~G. This threshold is defined as a parameter
(ZeroFieldThreshold) in forward/profiles.f90.

\subsection{The LINES file}
NICOLE needs to know the atomic line data of the transitions you want
to synthesize or invert. This is a configuration ASCII file in the ConfigObj standard
format. Comments are marked by the \# character. Everything following
a \# will be ignored. The file is divided in different sections. The
beginning of a new section is marked by a string enclosed in square
brackets ([ ]). In the case of the LINES file each section corresponds
to a spectral line. Inside a section we can have different parameters
that define the spectral line. Each parameter goes in a separate line
and the symbol = is used to separate the field from its value. For
example, a section defining a spectral line would look like this.

\begin{verbatim}
[FeI 6301.5]
   Element=Fe
   Ionization stage=1
   Wavelength=6301.5080
   Excitation potential= 29440.17 cm-1 # 3.65 eV
   Log(gf)=-0.59
   Term (lower)=5P2.0
   Term (upper)=5D2.0
   Collisions=Unsold
   Width=2
\end{verbatim}

The parsing is insensitive to case and indentation. The various
sections and the lines within each section can appear in any
order. Some fields are mandatory (wavelength, ionization stage, etc)
and others are optional (collisions, width, etc) and have default
values. Allowed fields in specifying a spectral line are the
following (mandatory unless noted otherwise):

\begin{itemize}
\item Element: Atomic element symbol (two characters). Case
  insensitive
\item Ionization stage: 1 for neutral, 2 for singly ionized or 3 for
  doubly ionized. Higher ionization stages are not supported
\item Wavelength: Central wavelength in \AA
\item Excitation potential: Can be given in units of eV (default) or
  cm$^{-1}$. To specify units, follow the numeric value with either eV
  or cm-1 (see example above)
\item Log(gf): Self-explanatory
\item Term (lower): A string of the form 2D1.5 (see also the example
  above). The first character must be a number with the lower level
  multiplicity (2s+1). The second character represents the orbital
  angular momentum. From the third character to the end of the string
  we specify the total angular momentum J (may be non-integer)
\item Term (upper): Same as above for the upper level
\item Collisions: (Optional, default=1). This is a flag indicating the
  treatment for collisional line broadening. Use 1 or the string
  Unsold to use the Unsold formalism (Unsold 1955). Set it to 2 or the
  string Barklem to use the approach of Barklem, Anstee and O'Mara
  (1998). This is the preferred option since it is more realistic, but
  it requires setting also the fields Damping sigma or Damping alpha
  which are line dependent. Note, however, that in this version of
  NICOLE collisions with neutral Helium are neglected compared to
  collisions with neutral Hydrogen. Normally this is a good
  approximation. Option 3 allows you to introduce manually the values
  of the radiative, Stark and van der Waals damping constants
  ($\gamma_r$, $\gamma_{Stark}$, $\gamma_{vdW}$). In this mode you can
  specify the additional fields {\bf Gamma Radiative}, {\bf Gamma
    Stark} and {\bf Gamma van der Waals} (see below). 
\item Damping sigma: (Optional, but mandatory if Collisions is set to
  2 or Barklem). Sigma parameter ($\sigma$) when using the
  Barklem et al formalism, in units of B\"ohr radius squared. 
\item Damping alpha: (Optional, but mandatory if Collisions is set to
  2 or Barklem). Alpha parameter ($\alpha$) when using the
  Barklem et al formalism. This parameter is dimensionless
\item Gamma Radiative: If {\bf Collisions} is set to 3, this sets the
  value of the damping constant $\gamma_r$ in units of 10$^8$ rad/s. The
  default would be the value obtained with {\bf Collisions = 1} (i.e.,
  using the Unsold formula).
\item Gamma Stark: If {\bf Collisions} is set to 3, this sets the
  value of the damping constant $\gamma_{Stark}$ in units of 10$^8$
  rad/s per 10$^{12}$ perturbers per cm$^{-3}$ at T=10,000~K. The code
  assumes a temperature dependence of T$^{0.17}$. The default would be
  the value obtained with {\bf Collisions = 1} (i.e., using the Unsold
  formula).
\item Gamma van der Waals: If {\bf Collisions} is set to 3, this sets
  the value of the damping constant $\gamma_{Stark}$ in units of 10$^8$
  rad/s per 10$^{16}$ perturbers per cm$^{-3}$ at T=10,000~K. The code
  assumes a temperature dependence of T$^{0.38}$. The default would be
  the value obtained with {\bf Collisions = 1} (i.e., using the Unsold
  formula).
\item Damping enhancement: (Optional, default=1). Additional multiplicative
  factor to apply to the collisional damping
\item Width: (Optional, default=2). Distance in \AA \, from line center
  at which the line has a significant opacity. This is used to speed
  up the calculation when a wide wavelength range is used. To be on
  the safe side this parameter should be set to a large value. For
  most photospheric lines, a value of 1 \AA \, is sufficient but for
  some strong chromospheric lines larger values are needed.
\item Mode: (Optional, default=LTE). Can be either LTE or NLTE. If the
  line is NLTE then some additional parameters are needed (see below).
\item Transition index in model atom: (Optinal, but needed if Mode is
  NLTE). Index of the transition in the model atom that corresponds
  with this line. For example if this line is the first transition in
  the model atom, set this value to 1.
\item Lower level population ratio: (Optional, default=1, used only
  when Mode is NLTE). Sometimes, the transition being referenced in
  the model atom is actually a multiplet and we are defining one of
  the lines of that multiplet. This parameter specifies the fraction
  $n_l \over n$ for the lower level, where $n_l$ is the population of
  the sublevel for the line we are defining and $n$ is the population
  of the level defined in the model atom.
\item Upper level population ratio: (Optional, default=1, used only
  when Mode is NLTE). Same as above for the upper level.
\end{itemize}
The values in the LINES file can be overriden by the NICOLE.input
file, as explained below. This is done so that one can have a
centralized database of spectral information and be able to make
temporary adjustments to the atomic parameters for the current run
without having to modify the central database. There is a sample LINES
file in the run/ directory of your distribution.

In case of conflict between parameters specified in the ATOM and LINES
(see below) files (e.g., the log({\em gf}) in LINES is not compatible
with the {\em g} and {\em F} parameters in ATOM), then the NLTE
iteration is done using the values in ATOM and the atomic populations
obtained are used for a final Stokes formal solution with the
parameters in LINES.

\subsection{The NICOLE.input file}
\label{nicinput}

There is a sample NICOLE.input file in your distribution.  This is
where we tell NICOLE exactly what we want to do. It is a
configuration ASCII file in the ConfigObj standard format. Comments
are marked by the \# character. Everything following a \# will be
ignored. The file may contain sections. The beginning of a new section
is marked by a string enclosed in square brackets ([ ]).

In each line the field label is separated from the value by the symbol
= (e.g., Mode=Synthesis). The parsing is insensitive to case and
indentation. If you don't remember exactly the field label, just write
your best guess. The Python wrapper run\_nicole.py will produce an
error if it encounters an incorrect line and will show one or more
helpful suggestions.

The various sections and the lines within each section
can appear in any order.  The main body of the file (before the
beginning of any sections) has the main parameters that control the
behavior of the code. Some of them are optional and have default
values. The following is a list of these parameters (mandatory unless
noted otherwise):

\begin{itemize}
\item Command: (Optional, default=../main/nicole). Starting in v2.0, the
  Fortran executable is not launched directly by the user. Instead, it
  is spanned by the Python script run\_nicole.py. Here you can specify
  what that command is. This is important because typically you need a
  more complicated form in order to launch the parallel version. For
  instance, using mpich2 to run on 8 processors, the command could
  look like this:
\begin{verbatim}
mpirun -n 8 ./nicole
\end{verbatim}
  In that case, we would specify:
\begin{verbatim}
  Command=mpirun -n 8 ./nicole
\end{verbatim}
  When supplying a string argument, as shown in this example, it is
  generally a good idea to enclose it within quotes to make sure that
  the parser interprets it as a single value. Otherwise, if we had for
  instance commas in the expression, it could be broken into a list of
  arguments. 
\begin{verbatim}
  Command='mpirun -n 8 ./nicole'
\end{verbatim}
\item Cycles: Number of cycles in the run (Optional, default=1). This
  needs to be specified either in NICOLE.input or
  NICOLE.input\_1. Each cycle is governed by a separate NICOLE.input
  file with the suffix \_n (where $n$ is the cycle number). If the
  file corresponding to a given cycle doesn't exist, NICOLE will try
  to use the values in NICOLE.input. Typically, if we wish to have a
  run of 3 cycles one sets Cycles=3 in NICOLE.input (or
  NICOLE.input\_1) and then create the files NICOLE.input\_2 and
  NICOLE.input\_3 with the parameters of the second and third cycles,
  respectively. {\bf Important: }Because of how the code is
  structured, all the input files required in any given cycle need to
  exist at the beginning of the run. So even if you use the output of
  one cycle to feed the input of another, or even if you use a file to
  serve simultaneously as input and output (both of which are
  legitimate strategies), you need to have at least a dummy file that
  exists and contains the right dimensions before you start the run. Also,
  all input files for cycles two and above must be in NICOLE's native format.
  This is because the format conversion is performed by the Python wrapper
  run\_nicole.py and it only works on the first cycle. 
\item Start cycle (Optional, default=1): You may want to skip one or
  more cycles. For instance, assume that your run was terminated
  halfway during the second cycle and you wish to resume it. You would
  need to use the Restart option (see below) and skip the first cycle
  when you rerun the code. Use this field to specify in which cycle
  you wish to begin execution. This field is only read in NICOLE.input
  or NICOLE.input\_1.
\item Mode: Can be the word Synthesis, Inversion or Convert (actually,
  only the first character is checked). Synthesis and inversion are
  self-explanatory. The third mode, convert is used to convert the
  geometrical height scale in a model to optical depth, or
  vice-versa. In this mode the code will take one of them (depending
  on the value of Height Scale described below), compute the other,
  write the output model file to disk and exit.
\item Input model: Name of the file containing the input model
  atmosphere. In synthesis mode, this is the atmosphere for which the
  spectral profiles are computed. In inversion mode this is the
  starting guess for the model. The file may contain one or many
  models. In synthesis mode, one set of Stokes I, Q, U and V is
  produced for each model. In inversion mode, if the number of models
  is smaller than the number of profiles then the last model is
  repeated to pad the calculation and ensure that all of the profiles
  are inverted. If the number of models is larger than the number of
  profiles, then the last models are ignored. For more information on
  the possible formats for this file, see section~\ref{modelfile}
\item Input model 2: Same as above but for the second component in
  case of a 2-component run. The filling factor, macroturbulence and
  stray light parameters in this model will be ignored. They are taken
  from the first model.
\item Output profiles: Name of the file that will be written by NICOLE
  with the output profiles. In synthesis mode this is the final result
  of the calculation. In inversion mode it contains the fits
  produced. For more information on the format of this file, see
  section~\ref{profilefile}.
\item Heliocentric angle: (Optional, default=1). Cosine of the
  heliocentric angle (usually denoted as $\mu$ in the literature).
\item Observed profiles (Optional, but mandatory if Mode is
  Inversion): Name of the file with the input observed profiles to
  invert. For more information on the format of this file, see
  section~\ref{profilefile}.
\item Restart: (Optional, deafult=0) Set to 1 to resume a previous run
  that was interrupted. See section~\ref{sec:restarting}. A value of
  -1 means to remove output files and then do normal run (not restarting
  previous calculation). Use this to make sure you don't accidentally mix
  older preexisting files with new results.
\item Output model: (Optional, but mandatory if Mode is Inversion).
  Name of the file with the output model atmosphere resulting from the
  inversion. In synthesis mode, the full model including gas pressure,
  density, electron pressure and optical depth scale (which might have
  been calculated internally by NICOLE) is written.  For more
  information on the possible formats for this file, see
  section~\ref{modelfile}
\item Output model 2: Same as above but for the second component in
  case of a 2-component run.
\item Formal solution method: (Optional, default=0). There are several
  formal solution methods for the Stokes radiative transfer equation
  currently implemented in NICOLE, including the Hermitian method of
  Bellot Rubio, Ruiz Cobo and Collados Vera (1998), the Weakly
  Polarizing Media (WPM) approximation described in Trujillo Bueno \&
  S\'anchez Almeida (1999), DELO (Rees et al 1989), Bezier splines (De
  la Cruz Rodr\'\i guez and Piscunov 2012) and short characteristics
  (Kunasz and Auer 1987). See the comments in the header of subroutine
  {\em formal\_solution} in forward/forward.f90 for details. The WPM
  approximation is faster, but not always applicable, while the
  Hermitian, DELO and Bezier methods are of general validity
  (provided, of course, that the spatial grid in the model atmosphere
  is fine enough). There is an automatic formal solver selector
  implemented in NICOLE, which will check at each wavelength whether
  WPM is suitable or not. If it is, it will be used. Otherwise, the
  cubic Bezier method is chosen. We strongly recommend you to leave
  this value set to 0 (auto).
\item Stray light file: (Optional). In real observations, especially
  when observing relatively dark structures such as sunspots, one
  normally has some amount of scattered light from the surrounding
  regions contaminating the observed signal. This stray light might
  come from scattering in the Earth atmosphere, from internal
  reflections in the telescope/instrument system, etc. If you would
  like to contaminate your synthetic spectrum with stray light, you
  must enter the stray light fraction in the model atmosphere file
  (see section~\ref{modelfile} below) and give here the name of a file
  with the stray light profile.  Leaving this field blank is
  equivalent to setting the amount of stray light to 0 in the model
  atmosphere and will result in no contamination of the synthetic
  profile. This profile can also be used as a prescribed external
  atmosphere that coexists within the spatial resolution element with
  the atmosphere undergoing synthesis or inversion. 
\item Printout detail: (Optional, default=1). This switch controls how
  much information is printed out to screen during normal
  operation. Higher values correspond to more detailed information
  (and more cluttering).  A value of 1 is normally a good choice.
\item Noise: (Optional, default=1e-3). Estimation of the noise in the
  observations (relative to the average disk center quiet Sun
  continuum intensity). This is used to compute the inversion weights
  so that a value of $\chi^2$=1 corresponds to a fit at the noise
  level. If the weights are supplied manually using a Weights.pro
  file, then this value has no effect.
\item Acceptable Chi-square: (Optional, default=0). This parameter is
  used to avoid (or at least minimize) the effects of local minima. If
  the inversion results in a $\chi^2$ value worse than this parameter,
  the code will discard this result and run again with a randomized
  initialization.
\item Maximum number of inversions: (Optional, default=5). How many
  inversions will be tried to reach the acceptable $\chi^2$ before
  giving up and picking the best result of the multiple inversion
  attempts.
\item Maximum inversion iterations: (Optional, default=25). Upper
  limit to how many iterations will be performed in an inversion
  cycle.
%\item Speed optimization: (Optional, default=1). Controls the behavior
%  of some approximations implemented to speed up the calculation. If 1
%  or more, use a newer vector routine for the calculation of Voigt
%  functions (recommended). If 2 or more, the code will use
%  neural networks for the conversion between gas and electron
%  pressure. This only works if the abundances employed are those of
%  Grevesse and Sauval (1998). IMPORTANT NOTE: This functionality is
%  currently broken but will be fixed in future versions.
\item Always compute derivatives: (Optional, default=Yes). If set to
  No, then the derivatives for the response functions are not
  recomputed after successful iteration steps. Only when an inversion
  step fails to improve the $\chi^2$ then the derivatives are
  recalculated. This will save some time and often works almost as
  good as when recalculating the derivatives.
\item Gravity: (Optional, default= 2.7414e+4). Surface gravity (in
  cm~s$^{-2}$). If not specified, the solar value is adopted.
\item Regularization: (Optional, default=1.0). During inversion
  operations, a regularization term is added to the $\chi^2$ function
  when the model has fluctuations or other generally undesirable
  behavior. In this manner, we establish a preference for well-behaved
  (e.g., smooth) models. This factor weights the regularization
  term. Setting it to 0 means no regularization at all. Higher values
  make the code care more about model smoothness than quality of the
  fit.
\item Update opacities every: (Optional, default=10). Background
  opacities don't change significantly over the narrow wavelength
  range spanned by a spectral line. This parameter specifies the
  wavelength intervals (in \AA ) at which the background opacities
  will be recomputed. Setting this to 0 makes the code recompute
  opacities for each wavelength point. Larger values avoid too
  frequent recalculations and therefore save some time.
\item Continuum reference: (Optional, default=1). Switch to control
  the normalization of the spectral profiles.  Select 0 for no
  normalization (output will be in c.g.s. units,
  e.g. erg~cm$^{-2}$~s$^{-1}$~Hz$^{-1}$ for flux and the same per
  strad for intensity). It is not recommended to use this option in
  inversion mode since the large values obtained (of the order of 1e14
  for average suiet Sun) will give rise to numerical precision issues,
  unless you use it in combination with the Continuum value field (see
  below) specifying a value of $\sim$1e14 to have the inverted data of
  the order of unity; 1 for HSRA continuum intensity {\bf at disk
    center} at a wavelength in the middle of each spectral range
  (default); 2 for normalization to HSRA continuum intensity {\bf at
    disk center} at 5000~\AA ; 3 for normalization to HSRA continuum
  intensity at a wavelength in the middle of each spectral range and
  at the heliocentric angle of the observations (spècified above); 4
  for {\bf local} normalization to the first point of each region 
  (useful to normalize to the local continuum) but note that this
  option eliminates all information on absolute photometry. Therefore
  the temperature scale retrieved is not physical.
\item Continuum value: (Optional, default=1). This is simply a
  normalization factor that will be applied to the input
  profiles. This is useful for instance if you are using data
  directly from an instrument in units of counts. You may want to
  divide by the number of counts of the average quiet Sun intensity to
  have a proper normalization (but it needs to be consistent with the
  Continuum reference parameter above).
\item Impose hydrostatic equilibrium: (Optional, default=Yes). If set
  to Yes, the electron pressure, gas pressure and density of the input
  model are re-evaluated to put the model in hydrostatic
  equilibrium. If No, then the values in the input model are used. In
  inversion mode, the model is always put in hydrostatic equilibrium.
  The hydrostatic equilibrium uses the electron pressure at the top of
  the atmosphere as a boundary condition, so at least that one value
  needs to be properly set. However, the stratification obtained is nearly
  insensitive to the boundary condition except at the higher layers.
\item Input density: (Optinal, default=Pel). The electron pressure,
  electron number density, gas pressure and gas density are related
  variables. Given one of them, the others can be determined
  univocally. This switch defines which one of the four is provided
  as input. The other two will be comptued by the code. Possible
  values for this field are Pel, Nel, Pgas or Dens. 
\item Height scale: (Optinal, default=Tau). The depth scale can be
  specified either as a geometrical scale (z in km) or as the
  continuum optical depth at 500 nm. Set this variable to either z or
  tau to use either scale. The other will be computed automatically by
  NICOLE from the temperature plus density or pressure (electron or
  gas). 
\item Keep {parameter} (Optional, default=0): Certain model parameters
  are computed internally by NICOLE by solving the equation of state,
  the ionization equilibrium and/or the molecular chemical
  equilibrium. It is possible to supply such parameters in the input
  model and instruct NICOLE to use those values instead of overriding
  them with its calculations. To do this set this field to 1 (this
  line can appear multiple times with different parameters). Possible
  accepted values for parameter are: Gas\_p, Rho, nH, nHminus, nHplus,
  nH2, nH2plus.
\item Eq of state (Optional, default=nicole): This is a switch to
  specify how to compute the electron pressure Pe from T and Pg, and
  conversely the gas pressure Pg from T and Pe. NICOLE has three
  different methods implemented (see Socas-Navarro et al
  2014). Possible values are: a)NICOLE to use our own method described
  in the paper above. This is probably the best compromise between
  speed, accuracy and stability; b)ANN to use artificial neural
  networks trained with precomputed values of (T,Pg,Pe). This is the
  fastest method but not as accurate as the others. c)WITTMANN to use
  the method described in Wittmann (1974), which in turn is an
  improvement over the procedure described by Mihalas (1967).
\item Eq of state for H (Optional, default=nicole): For the
  calculation of background opacities and some other quantities such
  as collisional rates, the code needs the relative distribution of
  some H states both in atomic and molecular form (the H ions H, H$^+$
  and H$^-$ and the molecules H$_2$ and H$_2^+$). There are several
  options for this switch: a)NICOLE to use the native method described
  in Socas-Navarro et al (2014). This is the default option and is
  probably the best compromise between speed, accuracy and stability;
  b)ASENSIO2 to use the chemical equilibrium method of Asensio Ramos
  (Asensio Ramos 2003) but restricted to only two molecules in the
  chemical equilibrium; c)ASENSIO273 to use the same method with the
  full list of 273 possible molecules; d)WITTMANN to solve for the H
  populations as in Wittmann (1974), which in turn is an improvement
  over the procedure described by Mihalas (1967).
\item Pe consistency (Optional, default=1e-3): Depending on how the
  equation of state is solved, it could be that the Pe, Pg values are
  not consistent due to the approximations employed. This means that
  if we take a pair (T,Pe) to compute Pg and later recompute Pe from
  this (T,Pg), the new Pg obtained will in general differ from the
  original one. If {\em Pe consistency} is set to a value lower than
  10, the Pe value obtained will be iterated until Pe and Pg are
  consistent within that tolerance. In principle this could slow down
  the procedure but in practice the difference in computing time
  should be negligible in almost all situations.
\item Opacity Package (Optional, default=Asensio): Package to use for
  the computation of background continuum opacities from most common
  contributors in the visible and infrared. Possible values:
  a)ASENSIO, b)WITTMANN, c)SOPA . This last option is hidden in the
  normal distribution because we haven't been able to adapt the
  original routines to conform with the standards. As such, it might
  cause problems with compilation and other issues. If there is any
  powerful reason why you feel that you'd need to use this package,
  please contact the authors for a SOPA-enabled version.
\item Opacity Package UV (Optional, default=TOP): Package to use for
  the computation of background continuum opacities from the most common
  contributors in the UV. Possible values: a)TOP, b)DM. The first
  option (TOP) will make use of photoionization cross-sections
  tabulated by The Opacity Project/The Iron Project for most neutral
  and singly ionized elements between Z=1 and Z=26. For Fe we use the
  data provided by Bautista (1997, A\&AS 122, 166) and by Nahar and
  Pradhan (1994, J. Phys. B 27, 429), using the smoothing technique of
  Allende Prieto (2008, Phys Scr 133). The second option (DM) uses the
  approximation of Dragon and Mutschlecner (1980, Apj 239, 104). These
  contributions are considered only at wavelenghts below 4000~\AA .
%\item Compute Hydrogen populations: (Optional, default=Yes). Some of
%  the underlying calculations depend (typically very weakly) on some
%  Hydrogen level populations. Normally, NICOLE will compute these
%  automatically assuming Saha-Boltzmann statistics but in some cases
%  the user might want to supply them directly. If this parameter is
%  set to No, the values in the input model will be preserved (unless
%  they are all zero, in which case the value of this flag will be
%  ignored). This flag is ignored whenever the hydrostatic equilibrium is 
%  recalculated. {\bf Note: This feature has been temporarily disabled
%  as of version 2.4, so effectively this parameter is set to yes always. 
%  The option to turn it off will be reinstated in the future.}
\item Start X position: (Optional, default=1). The model atmosphere and
  the observed profile files may contain many models or profiles. This
  parameter specifies at which position to start the calculation. It
  can be useful when resuming a previous run that was aborted for some
  reason. When working with a 3D cube you can specify a smaller field by 
  specifying also {\bf End X position}, {\bf Start Y position} and {\bf End Y position}.
  Pixel positions go from 1 to $n_x$ and/or $n_y$.
  Note that setting these values specifies a range of pixels to compute.
  If you don't want a smaller subfield but rather to restart the computation
  from a given point onwards, then use the {\bf Start irec} field explained
  below.
\item Start irec: (Optional, default=1). If a computation is interrupted for some reason and
  you wish to restart it at a given position, set this field to the
  restart pixel. If you are working with a datacube, keep in mind that
  irec=(x-1)$\times n_y$+y.
\item Debug mode: (Optional, default=0). Set this to a non-zero
  integer to switch on debug mode. In this mode, NICOLE will produce a
  core file with debugging information, useful to trace back any
  possible crash. One file is created by each process (in case of
  parallel run) at the start of each synthesis or inversion
  cycle. Upon completion, the file is removed.  Therefore, if a crash
  should occur you will end up with a file named core\_0.txt in your
  directory. In the case of the MPI version you'll get core\_1.txt,
  core\_2.txt ..., one for each process. To find out which process
  crashed, search for the string ABORTING in the core files (for
  example, in Unix use grep \"ABORTING...\" core*txt. Debug mode will
  slow down code execution slightly but not dramatically. You can use
  the procedure read\_debug in the idl/ directory to read the debug
  information. Some debug levels will make the code crash upon errors
  or warnings, others will make NICOLE reject the current iteration
  and try to recover. The precise meaning of each level can be
  found in the header of the main/debug.f90 source file.
\item Height scale: (Optional, default=tau). Set to z or tau to
  specify the height scale for the input model.
\item Optimize grid: Only for synthesis mode! If this switch is set,
  the model is reinterpolated to have a better vertical sampling (but
  keeping the same number of grid points) before solving the radiative
  transfer.
\end{itemize}

NICOLE.input can have an arbitrary number of sections defining the
various wavelength ranges that we want to operate on (whether in
synthesis or inversion mode). Each section starts with the label
[Region 1], [Region 2]... etc. Inside each region we define the
following parameters:
\begin{itemize}
\item First wavelength: Starting wavelength of the range in \AA .
\item Wavelength step: This is the sampling, can be given in \AA
  (default) or in m\AA . For the latter, simply follow the number with
  the string mA. 
\item Number of wavelengths: Self-explanatory
\item Macroturbulent enhancement: (Optional,
  default=1). Macroturbulence is specified as a velocity and therefore
  it scales linearly with wavelength. However, the spectral resolution
  delivered by a spectrograph instrument may vary from one region to
  another depending on the order employed and other specifics of the
  configuration. This parameter multiplies the width of the
  macroturbulence Gaussian in the current region.
\end{itemize}

At least one region must be defined. NICOLE.input also contains
definitions of the lines to synthesize/invert. These are specified in
separate sections. Each one of these sections start with the label
[Line 1], [Line 2], ... etc. Inside each one of these sections we
specify the parameter Line with the identification of the line in the
LINES database file. For instance:

\vskip12pt
\leftline{
Line=FeI 6301.5}
\vskip12pt

The line must be defined in LINES. In this section we can also
override the values given in LINES simply by specifying again the
parameter with a different value. At least one line must be defined
(for continua calculation, set the Log(gf) to a very small value)

When doing inversions, the number of nodes to be used must be
specified in a section of the configuration file. It starts with the
line:

\vskip12pt
\leftline{
[Nodes]}
\vskip12pt

In this section we specify the number of nodes. Valid tokens are:
Temperature, T, Velocity, Microturbulence,
Macroturbulence, Bz, Bx, By,
Stray light, Filling factor, Abundances.
Here is an example of the nodes section:

\begin{verbatim}
[Nodes]
# Nodes for first atmospheric component
#   Commented in parenthesis are default values for different cycles
Temperature=4       # (4, 8, 10)
Velocity=1          # (1, 4, 6)
Bz=1                # (1, 4, 4)
Bx=1                # (1, 2, 2)
By=1                # (1, 2, 2)
Microturbulence=0   # (0, 2, 2)
Macroturbulence=0   # (0, 1, 1)
Abundances=0        # (0, 0, 0)
Stray Light=0       # (0, 1, 1)
Filling Factor=0    # (0, 0, 0)
\end{verbatim}

By default NICOLE will set the nodes equispaced through the
atmosphere. You may override this and explicitly set the location of
the nodes by supplying a comma-separated list of heights. The whole list must be enclosed within quotes ("). For instance:
\begin{verbatim}
[Nodes]
Temperature=" -5.5 , -4 , -3.5 , -2 , 0 , 2"
\end{verbatim}

For NLTE calculations we can include a section to override the default
parameters that control the NLTE iteration, convergence, etc:
\begin{itemize}
\item Elim: (Optional, default=1E-3 for synthesis, 1E-4 for inversions). The
NLTE iteration will stop when the maximum relative change in the
atomic level populations, considering all levels and all depth-points,
is below this value.
\item isum: (Optional, default=1). Which statistical equilibrium
  equation will be replaced by the particle conservation equation to
  close the system.
\item istart: (Optional, default=1). Switch to control the initial
  guess for the atomic level populations. 0: Initialize the solution
  of the statistical equilibrium for zero radiation field; 1: Start
  with LTE populations.
\item CPER: (Optional, default=1.0). Artificially enhance collisional
  rates by this factor.
\item Use collisional switching: (Optional, default='n'). If set to
  'yes' use the collisional switching scheme, in which the NLTE
  iteration is started with a high value for CPER which is then
  gradually decreased as the iteration progresses until it finally
  becomes unity.
\item NMU: (Optional, default=3). Number of points in the angular quadrature.
\item QNORM: (Optional, default=10). Wavelength normalization value in km/s.
\item Formal Solution: (Optional, default=1). Switch to control the
  formal solution routine to be employed. Set to 1 to use Delo bezier
  splines or 2 for short characteristics.
\item Linear formal solution: (Optional, default=0). If set to 1
  forces the solution of the radiative transfer in the NLTE module to
  use the linear approximation. This is usually more stable but somewhat
  less accurate than the default.
\item Optically thin: (Optional, default=1e-3). Monochromatic optical
  depth above which we consider the atmosphere to be transparent.
\item Optically thick: (Optional, default=1e3). Monochromatic optical
  depth below which we consider the atmosphere to be thick and use the
  diffusion approximation (but neglecting the source function gradient)
\item Vel Free: (Optional, default='y'). Whether or not to use the
  velocity free approximation.
\item NGACC: (Optional, default='y'). Whether or not to use NG acceleration.
\item Max Iters: (Optional, default=500). Maximum number of allowed
  interations in the NLTE computation. 
\item Lambda Iterations: (Optional, default=3). Number of lambda iterations to perform at the beginning of the NLTE computation.
\item Ltepop: (Optional, default='nicole'). Set to either 'multi' or
  'nicole' to switch between two different approaches in the
  calculation of the LTE populations. In the MULTI approach, the sum
  of statistical weights that appears in the Saha equation is done
  over the finite number of levels considered in the model atom. In
  the NICOLE approach, tabulated partition functions are used to
  determine an approximate sum over an infinite number of levels.
\item Elements to ignore in backgroudn opacities: (Optional,
  default=''). If the NLTE calculation includes the bound-free
  transitions treated in detail, we might want to exclude that element
  in the calculation of background opacities. A typical example is
  the H atom. All background opacity packages in NICOLE consider the
  opacity produced by bound-free transitions of neutral hydrogen. If
  we are computing a H atom that includes those transitions then it's
  a good idea to put H in this field. The background opacity package
  would then skip the computation of opacities due to neutral H
  bound-free transitions.
\end{itemize}

Finally, we can optionally have a section about abundances in
NICOLE.input. It is defined in its own section, which starts with the
line:

\vskip12pt
\leftline{
[Abundances]}
\vskip12pt

This section can have the following optional parameters:
\begin{itemize}
\item Abundance set: (Optional,
  default=grevesse\_sauval\_1998). NICOLE has several abundance sets
  preset in the code (actually, they are in the Python
  wrapper). Current options are: Grevesse\_Sauval\_1998,
  Asplund\_et\_al\_2009, Thevenin\_1989, or Grevesse\_1984
\item Abundance file: (Optional). If present this is the name of an
  ASCII file containing the abundances for the 92 first elements. Each
  line has an element, defined by its two-letter symbol followed by
  the = sign and its abundance on the usual log scale that has H as
  12. For instance

\vskip12pt
\leftline{
H=12}
\vskip12pt
\end{itemize}

Inside this Abundances section we can have a subsection named
[[Override]] in which we can specify a discrete set of elements for
which we want to override the default values (whether from the
hardwired databases or from the external file). This is done simply by
including the element to override in this subsection. A full example
would be:

\begin{verbatim}
[Abundances]
Abundance set=Thevenin_1989
[[Override]]
O=600 ppm
He=0.095
\end{verbatim}

In the override subsection, the abundances can be specified also in
parts per million simply by following the number with the ppm string

\subsection{The model atmosphere file}
\label{modelfile}

This file contains one or more model atmospheres for which the
synthetic spectrum will be computed. Units for the various physical
variables are c.g.s. (K for temperature, g cm$^{-3}$ for density, dyn
cm$^{-2}$ for pressure, cm s$^{-1}$ for velocity, G for field strength
and degrees for field inclination and azimuth) except for z which is
height in km (positive values correspond to higher layers). The
line-of-sight velocity sign follows the Astrophysical convention where
positive values represent downflows (redshift). The magnetic field in
this version is defined primarily by its components with respect to
the line of sight. B\_long represents the longitudinal (along the line
of sight) component, whereas B\_x and B\_y are the two components on
the plane of the sky. The coordinates x and y are arbitrary but they
define the reference frame for the linear polarization Stokes Q and
U. The ``local'' variables refer to the solar reference frame. The
inclination is defined with respect to the vertical, so 0 is field
pointing up, 90 is horizontal and 180 is pointing down. The azimuth
reference is arbitrary but the absorption profile for linear
polarization is such that Q is positive and U is zero when the azimuth
is 0 or 180.

The file can be in one of several formats:

\subsubsection{ASCII}

This is the same format used in versions prior to 1.6. Have a look at
the HSRA.model file included in your distribution.  Any line starting
with a ! symbol is a comment and will be ignored.  The first
non-comment line in this file must be the string:

\vskip12pt
\leftline{
Format version: 1.0}
\vskip12pt

This is an internal identifier which is used for backwards
compatibility.  Don't change that line or NICOLE will complain about
it and refuse to work.  The second non-comment line contains two or
three real numbers. The first one is the macroturbulent velocity, in
cm/s. After computing the synthetic profile, it will be convolved with
a Gaussian whose half-width is this value. However, note that if a
file named Instrumental\_profile.dat exists in the running directory,
then the macroturbulence will be ignored (see \ref{instprof}
below). The second 
number in this line is the fraction of stray light that will be added
to the synthetic profile. It must be in the range [0,1]. Note that the
stray light parameter may also be used to account for a magnetic
filling factor. 

The following lines describe the depth-dependence of the model.
NICOLE will read eight columns, which stand for log($\tau_{5000}$),
temperature (in K), electron pressure (in dyn/cm$^2$), microturbulence
(in cm/s), longitudinal magnetic field (in Gauss), line-of-sight
velocity (in cm/s), transverse field (i.e., on the plane of the sky)
on the x direction (in Gauss) and transverse field on the y direction
(in Gauss), respectively.  The log($\tau_{5000}$) scale does not need
to be equispaced and it may be either increasing or decreasing, but it
must be monotonic. Instead of electron pressure, the third column may
be electron number density, gas pressure or density (all in c.g.s
units), depending on what has been specified in the parameter Input
Density in NICOLE.input

\subsubsection{IDL savefile}

We can use an IDL savefile as inputs. The file may contain the
following variables: z, tau, t, el\_p, gas\_p, rho, v\_los, v\_mic,
b\_los\_x, b\_los\_y, b\_los\_z, b\_x, b\_y, b\_z, nH, nH$^-$, nH$^+$,
nH$^2$, nH$^{2+}$ as arrays with dimensions (nx, ny,
nz). Additionally, the following (nx,ny) arrays might exist:
keep\_el\_p, keep\_gas\_p, keep\_rho, keep\_nH, keep\_nHminus,
keep\_nHplus, keep\_nH2, keep\_nH2plus, v\_mac, stray\_frac,
ffactor. Finally, the array abundance (nx,ny,92) can be used to
specify the chemical composition of the atmosphere, in Astrophysical
logarithmic units. All variables are in c.g.s. units (tau actually
refers to the base-10 logarithm of the optical depth at 500nm). The
keep\_xxx variables are actually flags (even though they are defined
as real*8 numbers to maintain the same datatype in the entire
model). Each flags specifies whether NICOLE should compute that
variable internally (keep\_xxx=0, the default behavior) or use the
value supplied in the model (keep\_xxx=1).

Sometimes, when creating or manipulating arrays where one of the
dimensions has only 1 element, IDL will supress such dimension. Before
you write the IDL savefile with the data, please make sure that the
model variables have three dimensions (or two in the case of v\_mac,
stray\_frac, ffactor, and the keep flags). If you are having trouble
with the array dimensions you can also include nx, ny and nz in the
IDL savefile and NICOLE will use them to interpret the arrays
correctly. As a last resort, if the arrays are not three-dimensional
and nx, ny or nz are not properly specified, NICOLE will try to
interpret the missing dimensions as being 1 (starting with z, then y
and finally x).

\subsubsection{NICOLE native format}

NICOLE works internally with binary direct-access little-endian
files. This is true even if you are running on a big-endian
machine. NICOLE recognizes the endianness of the machine and, if
running on big endian, will do byte swapping before reading and
writing to disk. Output models produced by the code are in this format.

The record size for model files is 22*nz+11+92 real numbers of kind 8
(meaning, 8 bytes per number). Each record corresponds to one model
atmosphere and has all variables stored sequentially in the following
order: z(1:nz), log tau\_500(1:nz), t(1:nz), gas\_p(1:nz), rho(1:nz),
el\_p(1:nz), v\_los(1:nz), v\_mic(1:nz), b\_long(1:nz), b\_x(1:nz),
b\_y(1:nz), b\_local\_x(1:nz), b\_local\_y(1:nz), b\_local\_z(1:nz),
v\_local\_x(1:nz), v\_local\_y(1:nz), v\_local\_z(1:nz), nH(1:nz),
nHminus(1:nz), nHplus(1:nz), nH2(1:nz), nH2plus(1:nz), v\_mac,
stray\_frac, ffactor, keep\_el\_p, keep\_gas\_p, keep\_rho, 
keep\_nH, keep\_nHminus, keep\_nHplus, keep\_nH2, keep\_nH2plus,
abund(1:92). The ``local'' b and v represent
the magnetic field and velocity vectors in the local solar frame of
reference.  These parameters are not used by NICOLE directly. They are
used by the incline.py program to pre-process a 3D model cube,
transforming it from vertical to an inclined line-of-sight. During
this transformation, the local b and v are used to compute b\_long,
b\_x, b\_y and v\_los, which are the variables needed by NICOLE. In
summary, if you use incline.py, then b\_local and v\_local are used
but v\_los, b\_long, b\_x and b\_y are ignored. If you don't use
incline.py then it's the other way around.

 The keep\_xxx variables are actually flags (even though they are
 defined as real*8 numbers to maintain the same datatype in the entire
 model). Each flags specifies whether NICOLE should compute that
 variable internally (keep\_xxx=0, the default behavior) or use the
 value supplied in the model (keep\_xxx=1).

The model file contains npix+1 records (npix being the number of models in
the file, equal to nx$\times$ny). The first record is actually a
signature to allow the code to recognize the file and also to provide
the nx, ny and nz parameters needed to dimension the variables. This
signature has the following format. The first 11 bytes contain the
string nicole2.3bm. The following 5 bytes are 0s, and then come two
32-bit integers representing nx and ny and one 64-bit integer with
nz. From there on the record is padded with zeros.

\subsection{The instrumental profile file (optional)}
\label{instprof}

The instrumental profile is the response of the instrument to a
monochromatic beam. Often this is modeled with the macroturbulence
assuming that the instrumental profile is a Gaussian but sometimes
this approach does not suffice. If we know the instrumental profile of
our instrument (e.g., from theoretical considerations or by measuring
the spectrum produced when the system is illuminated by a laser beam)
we can specify it by including an ASCII file in the running directory
named Instrumental\_profile.dat. One can also use the suffix \_1, \_2,
etc to have a different file for each cycle of the run. If the file
exists, NICOLE will read and use it. Make sure you have spelled the
name correctly, though. If the Printout Detail parameter in
NICOLE.input is 1 or higher, NICOLE will print the following message:
\begin{verbatim}
 Reading Instrumental_profile.dat
\end{verbatim}
The format of this file is the same as any other profile file. This
has changed from versions prior to 2.6 to allow for an instrumental
profile that varies over the field of view, as is the case with some
instruments. Note that each profile might contain different spectral
regions. We then need to include the appropriate profile for each
region in each profile. The instrumental profile needs to have its
peak at the first point, then have the right-hand side of the profile
function run through half of the region profile and, finally, from the
midpoint to the last we have the left side of the profile. If the
instrumental profile is specified, macroturbulence is ignored. It is
recommended to test this option with a simple synthesis of a narrow
line (without micro- or macroturbulence), observing the broadening
produced when the instrumental profile is used.


\subsection{The departure coefficients file (optional)}
\label{depcoef}

Unless you really know what you are doing, you can (and probably
should) skip this section.  We can specify departure coefficients that
will be applied by multiplying the opacity (lower level departure
coefficient) and emissivity (upper level departure coefficient) of the
lines computed by NICOLE. This could be potentially useful to apply
{\em ad-hoc} NLTE corrections to the lines. To do so you need to have
a file named depcoef.dat. The file is in binary format with a record
length of $2\times nl \times nz$ reals of kind=8, where $nl$ is the
number of lines and $nz$ the number of vertical grid points in the
model atmosphere. For each $x$ pixel the file has a record
that contains for each line the departure coefficients for the lower
level at all depth-points and then similarly for the upper level. 

Note that the departure coefficients need to be in the same grid as
the model atmosphere. Also, it is assumed that the coefficients are
ordered starting at the top of the atmosphere. If this file is read
and the departure coefficients are used, NICOLE will print the
following message (assuming that Printout Detail in NICOLE.input is 1
or higher):
\begin{verbatim}
 Reading depcoef.dat
\end{verbatim}

There is an alternative use of the depcoef.dat file in which you can
supply actual populations directly (actually, one introduces the ratio
$n/g$ which is what the code will use internally). To do this, introduce 
the following line in NICOLE.input:
\begin{verbatim}
Depcoef behavior= 2
\end{verbatim}
and for each level write the populations divided by the statistical weight
($g$) as explained above (remember: starting from the top of the atmosphere).

\subsection{Running NICOLE in synthesis mode}
\label{sec:runsyn}

Ok, so now that we know all the input files, what they do and how they 
are written, we only have to run the Python wrapper:

\vskip12pt
\leftline{
./run\_nicole.py}
\vskip12pt

at your command prompt. This program, more than a wrapper, is almost an
entire code by itself. It creates the input files for the
Fortran code and runs it. The input files created by run\_nicole.py
start with a double underscore (e.g., \_\_inputmodel.bin). To avoid
accidentally overwriting your own files, it is strongly recommended that
you do not use any files starting with a double underscore and leave
this prefix for NICOLE and its support programs.

You have to be aware that, if the output files already exist, NICOLE
will not replace those files. Instead, it will overwrite the records
that it has computed in the present run. Other preexisting records in
the file will be preserved. This can be good in some situations but
bad in others. For example, NICOLE has a ``Restart'' mode in which it
is possible to resume a run that was interrupted before completion. Or
sometimes one uses the input model file with the guess model
atmosphere to contain also the atmospheres produced by the
inversion. In these situations it is good to have the ability to use
existing output files. However, there may also be situations in which
one inadvertently overwrites parts of a preexisting files and ends up
with unexpected results. If a given output file exists before the run,
NICOLE will issue a warning but proceed anyway. If you want to make
sure that your run is fresh new, set Restart to -1 (see
Section~\ref{nicinp} above) and the output files will be removed at
the beginning of the run.

When you run the code you may get some warning messages about the line
not being optically thin at the surface. This is because the model
does not extend high enough. In any case, the values for $\tau_\nu$
that NICOLE is reporting are $\sim 10^{-2}$, which is good enough for
our purposes here, so don't worry too much about it. Ok, so everything
worked fine and now you have a new file in your directory with the
synthetic data.  Congratulations! You have run your first calculation
with NICOLE!

If you specified an output model file in NICOLE.input, NICOLE will
output the actual model that it used to do the synthesis. This might
not coincide exactly with the model you supplied due to a number of
reasons. First of all, if you have the optimize grid option set,
NICOLE has reinterpolated your model to a more suitable
grid. Moreover, some of the variables defining the plasma state are
computed by NICOLE's equation of state routines, possibly overriding any values
you might have supplied. For instance, if you set input density to
electron pressure, all variables will be recomputed from T and Pe:
density, gas pressure and number densities of neutral hydrogen,
ionized hydrogen, negative hydrogen ion, hydrogen molecule and singly
ionized hydrogen molecule. You can instruct NICOLE to retain the
original values you supplied in the input model for any of these
parameters by setting the various keep flags, as explained in
section~\ref{nicinput}. So, for instance, if you wish to use the gas
density, gas pressure and neutral hydrogen density provided in your
input file, you would need to set {\em Keep Rho}, {\em Keep Gas\_p}
and {\em Keep nH} to 1 in NICOLE.input (see section~\ref{nicinput}).

If the NLTE inversion does not converge (to see this you may need to
have the printout parameter set to 3 or greater in NICOLE.input) try
playing around with the parameters in the NLTE section of
NICOLE.input. The first thing to try would be to use linear
interpolation by setting {\em linear formal solution} to 1. You can
then try changing the formal solution from 1 to 2 or viceversa.

The format of the file that you have just obtained, HSRA.pro, is
explained in section~\ref{profilefile} below. You can read it in IDL
with the read\_profile.pro procedure. On exit, NICOLE produces a file
with profiling information, named profiling\_\_n.txt, where n is the 
process number. In case of the serial build there is only one file but
the parallel build produces one file per process. This file shows the
code execution time and also breaks down how this time is spent inside
the most time-consuming routines. Note that some of the routines
profiled are actually called by others. For example, solvestat is
called by forward, which in turn is called by
compute\_dchisq\_dx. Because of this the percentages don't add up to
100\%.

\subsection{The profile file}
\label{profilefile}

This file contains one or more spectral profiles (observed or
synthetic). The file can be in one of several formats:

\subsubsection{ASCII}

This file is arranged in five columns and has as many rows as
wavelengths in your spectrum.  The first column contains the
wavelength in \AA . Columns two through five contain the Stokes I, Q,
U and V parameters, respectively. The Stokes parameters are normalized
according to the corresponding setting in NICOLE.input

\subsubsection{IDL savefile}

We can use an IDL savefile as inputs. The file must contain the
following variables defined: stki, stkq, stku and stkv. They must be
arrays of 3 dimensions: nx, ny and n$\lambda$
(where n$\lambda$ is the number of wavelengths in the wavelength
grid). If several regions are defined then the profiles for different
regions are listed sequentially for each x, y point.

Sometimes, when creating or manipulating arrays where one of the
dimensions has only 1 element, IDL will supress such dimension. Before
you write the IDL savefile with the data, please make sure that all
stki, stkq, stku and stkv have three dimensions. If you are having
trouble with the array dimensions you can also include nx, ny and
n$\lambda$ in the IDL savefile and NICOLE will use them to interpret
the arrays correctly. As a last resort, if the arrays are not
three-dimensional and nx, ny or nz are not properly specified, NICOLE
will try to interpret the missing dimensions as being 1 (starting with
n$\lambda$, then y and finally x).

\subsubsection{NICOLE native format}

NICOLE works internally with binary direct-access little-endian
files. This is true even if you are running on a big-endian
machine. NICOLE recognizes the endianness of the machine and, if
running on big endian, will do byte swapping before reading and
writing to disk. Output models produced by the code are in this format.

The record size for profile files is 4*n$\lambda$ real numbers of kind 8
(meaning, 8 bytes per number). Each record corresponds to one set of
the 4 Stokes profiles and has all variables stored sequentially in the
following order: I($\lambda_1$), Q($\lambda_1$), U($\lambda_1$),
V($\lambda_1$), I($\lambda_2$), Q($\lambda_2$), U($\lambda_2$),
V($\lambda_2$), ... ,I($\lambda_{n\lambda}$), Q($\lambda_{n\lambda}$),
U($\lambda_{n\lambda}$), V($\lambda_{n\lambda}$)

The file contains npix+1 records (npix being the number of profiles in
the file, equal to nx$\times$ny). The first record is actually a
signature to allow the code to recognize the file and also to provide
the nx, ny, n$\lambda$ parameters needed to dimension the
variables. This signature has the following format. The first 11 bytes
contain the string nicole2.3bp, followed by 5 bytes with 0s. Then come
two 32-bit integers representing nx and ny, and one 64-bit integer
with n$\lambda$. From there on the record is padded with zeros.


\section{The inversion mode}
\label{invmode}

\subsection{Setting the input parameters}

At this point you already know almost everything you need to run
NICOLE. In this section you will learn how to use the code in
inversion mode. First, we have to change the operation mode in the
NICOLE.input file. Edit this file and set Mode to ``Inversion''. Now
the meaning of the fields Input Model and Synthetic Profiles is
different. The input model is a starting guess model, and the
synthetic profiles are the profiles emergent from the retrieved model
that we will obtain as a result of the inversion. So set input model
to ``guess.mod'' and synthetic profiles to ``modelout.pro''.

In addition, there are two new fields that we must complete in
inversion mode. The Observed Profiles field must be set to the name of
a file with the observed profiles that we wish to invert. Let us test
the code by using the HSRA.pro profiles that we synthesized above as
observed profiles. This way we can check whether or not the retrieved
model is similar to the HSRA.model. The other field that we must
complete is Output Model which should be set to the name of the file
that will contain the model obtained as a result of the inversion. In
this example, set it to ``modelout.mod''.


\subsection{Running NICOLE in inversion mode}

That's all we need to do. Now run NICOLE exactly as in
section~\ref{synthmode} and you will notice some differences. She will
take a little longer to run now and will display messages like this:

\vskip12pt
\leftline{
iter=1 Lambda=10. Chisq=3941.23901}
\vskip12pt

Each time you see one of these messages, NICOLE has performed a
successful inversion iteration step.  The numbers in this line have
the following meaning. First we have an integer number, iter, which
shows how many iterations have been performed so far.  Then we have
the diagonal element $\lambda$, which is a parameter used in the
Levenberg-Marquardt algorithm (you don't need to worry about that for
the moment). Finally, Chisq is the merit function $\chi^2$, which
measures the quality of the fit.  A nice fit is obtained when $\chi^2$
is around or below one.  Obviously, the fit is still poor in the first
iteration.  But no sweat. It will improve quickly. However, the actual
values of $\chi^2$ depend on the weights employed.

Instead of the message above, you may read something like 
this from time to time:

\vskip12pt 
\leftline{
REJECTED:--- iter=9 Lambda=0.100000001 Chisq=0.607707798}
\vskip12pt

This means that the model proposed by NICOLE in this iteration does
not yield an improvement to the current value of $\chi^2$, and
therefore it has been rejected. After three successive failed
attempts, NICOLE will quit. If, at this point, the fit is not yet
satisfactory, NICOLE will add more degrees of freedom to the model
(e.g., will allow for gradients in the magnetic field strength and
orientation, more complicated velocity and temperature
stratifications, etc.)  and restart a new set of iterations from the
current guess model.


Another warning message that you might occasionally get is the following:

\vskip12pt
\leftline{
Clipping temperature}
\vskip12pt

This means that the model proposed by NICOLE has exceeded the allowed 
temperature range at one or more depth-points. When that happens, 
she will bring the model back within range by performing a linear 
transformation. Normally this will not represent any problem,
and it is not something you should worry about. The other 
atmospheric parameters are also monitored and clipped when they get 
out of range, so you might get similar warnings regarding the magnetic 
field, the fraction of stray light, etc.

Once the inversion is finished you can check how good it was by
reading the profiles and models with the read\_profile.pro and
read\_model.pro IDL procedures.

Congratulations! You have successfully inverted your first spectrum
with NICOLE.

\subsection{Inversion weights}

It is often useful to give different weights to the different Stokes
profiles, because the amplitudes of I, Q, U and V typically differ in
one or two orders of magnitude. NICOLE uses an automatic weighting
scheme defined in the compute\_weights routine (which can be found in
the misc/ directory of your source code distribution). This scheme
takes into account the different amplitudes of the profiles and the
noise of the observations. However, one may want to fine tune or
customize this scheme. Another reason to weight the profiles is to
discard telluric lines or any other contamination that may be present
in the observations.

The default weights can be overridden by creating a file named
Weights.pro in the working directory. This file must have the same
format as a regular ASCII profile file (see
section~\ref{profilefile}), but contains the $\sigma^2$ values for I,
Q, U and V at each wavelength point. The actual weight is inversely
proportional to $\sigma$, so for instance if one wants to ignore a
given wavelength range, we just set $\sigma^2$ to a very large value
in this file. Another way to override weights is to set some points in
the input value to negative values beyond -10 (meaning larger absolute
values). 

Whether one uses default or custom weights, NICOLE will produce an
output file Weights\_used.pro with the weights it has used in ASCII
format.


\subsection{Changing the default number of nodes}

Advanced users may want to change the way NICOLE selects nodes for the
inversion. This is done in the NICOLE.input\_n files (see section~\ref{nicinput}). 


\subsection{Monitoring the inversion}
After each successful iteration, NICOLE will output the current guess 
model and the emergent profiles in the files tmp.mod and tmp.pro. 
If you would like to monitor the progress of the
inversion in real time you can plot these files at any time.
A tmp.err file with the error bars is also output at each iteration
(see section~\ref{errorbars} below).


\subsection{The error bars}
\label{errorbars}

NICOLE will output the error bars for the current guess model as tmp.err, 
and for the final model in a file named as the output file, 
but with extension ``.err''. Basically, the errors file
has the same format as a model atmosphere file. 
All the physical variables are set to $-1$ at all the depth-points, 
except at those points where we have ``nodes''
of the inversion, where the corresponding errors are given.

The error bars must be interpreted with care. First of all, they are
absolutely meaningless until the minimum of $\chi^2$ is reached, which
means that one shouldn't take the error bars seriously until the
inversion is done, and only when a minimum has been reached. Moreover,
you must understand that these bars provide the uncertainties on the
retrieved free parameters, so they should be interpreted in the
following manner. NICOLE will assume that the model sought can be
constructed from the starting guess model plus a correction to be
determined during the inversion. This correction may be a constant
value, a straight line, a parabola, or a higher-degree polynomial,
depending on how much freedom is given to a particular physical
parameter. For example, NICOLE will start considering a parabolic
correction to the temperature (which means three inversion nodes), a
linear correction to the l.o.s.\ velocity (two inversion nodes), and a
constant (depth-independent) correction to the magnetic field vector
and to the microturbulence (only one node). Then, for example, the
three error bars that we will obtain for the temperature define the
range of parabolas that are compatible with the observations.


\subsection{The file maskinvert.dat}

This file is optional. If it exists, it specifies which points should
be inverted and which ones should not. There is one real number per
record. A value of zero signals the code to ignore this $(ix, iy)$
profile. A value of one signals the code to perform the inversion.


\subsection{Tips for successful inversions}
\label{sec:tips}

\begin{itemize}
\item Initialize with physically sensible models. 
      Since NICOLE will start with constant corrections 
      to the magnetic field vector and microturbulence, 
      and with linear corrections to the l.o.s.\ velocity, 
      it is a good idea to initialize with models having 
      a constant magnetic field and a constant or 
      linear l.o.s.\ velocity.
\item If it doesn't work the first time, try with a different 
      initialization. If you see clipping warnings before and 
      in between the rejected iterations, that might be signaling that
      NICOLE needs to perform corrections that are pushing the model out 
      of the allowed range.
      Take the tmp.mod model, simplify the depth-dependence of the 
      clipped quantity and use it as a new 
      initialization.
\item Sometimes it is a good idea to take the result of an inversion, 
      perturb some of the physical variables, and use it as initialization 
      for a new inversion. You may obtain better fits.
\item If you are having problems, check the troubleshooting chapter. 
      You may find useful information there. If everything else fails 
      or you think you have found a bug in the code, please send email to
      \mbox{hsocas@iac.es}. 
\end{itemize}


\subsection{Restarting an inversion}
\label{sec:restarting}

If you're running an inversion with many points and your computation
is interrupted, you can set the Restart option to 'Y' in NICOLE.input and
then simply restart the code. It will look at the output files
(Chisq.dat, as well as profile and model output) and skip those points
with a $\chi^2$ value that is better than the Acceptable Chi-square
field in NICOLE.input. If your computer uses a disk cache (and
nowadays pretty much they all do), you will lose whatever results were
not physically written to file. One way to avoid this is to
include Flush statements to make sure the files get written after each
inversion. Unfortunately, Flush is not part of the Fortran 90
standard. When creating the makefile, the create\_makefile.py script
(see next section) will check if Flush is available in your platform
and include it in the source code if/where appropriate.

In order to restart an inversion using the previous data, you simply need 
to include the following option in your NICOLE.input file.

\begin{verbatim}
Restart= Yes
\end{verbatim}

You may also want to set one of Start irec, Start X position or Start Y
position.

\subsection{Debugging and profiling}

NICOLE includes a set of tools to help identify and fix problems. We
can activate debug mode in NICOLE.input simply by including this line:
\begin{verbatim}
Debug mode=1
\end{verbatim}
Two things happen in this mode. First, a number of files are created
with information that allows one to trace back the causes of a problem, 
especially a crash. One file is created for each parallel process. They
are named core\_n.txt (where n identifies the process that created it).
These files are simply ASCII text files with debug information. For
each pixel, after successfully completing a synthesis or an inversion
cycle, the file is destroyed. When a crash happens, the core files are
left behind with the current code status. Make a note of what process
caused the crash (this is usually printed out by the OS) so you know
which core file you need to analyze. The second thing that happens in
debug mode is that a number of runtime checks are performed to verify
the sanity of the current run. When debug mode=1, an abnormal condition
results in a wealth of information printed out to the console but the code
will continue executing, hoping to recover from this condition. Setting 
debug mode to a value larger than 1 signals the code to stop upon 
encountering such abnormal situations
so that you can then inspect the core files produced. The idl directory
contains a procedure (read\_debug.pro) that can be useful to inspect
core files. If you have multiple processes, and therefore multiple
core files, you'll need to specify which one to read upon invocation
of read\_debug.pro. The procedure returns a structure with the most
interesting variables at the time of the crash and the current model
atmosphere being used by the code.

In addition to the debug mode, we can set a number of other flags in
NICOLE.input to output some further information. Do NOT use these options
in MPI mode:
\begin{itemize}
\item {\bf Output populations:} At the end of each synthesis, create a
  binary file named Populations.dat. This file is overwritten after
  each synthesis, including those that are part of the inversion
  process. It doesn't work in parallel mode, as all processes would be
  writing concurrently in the same file. The file has binary format
  and has record length nz. If you are doing a NLTE calculation the
  file contains the populations of all the atomic levels (one record
  corresponds to one level). After the converged level populations it
  contains the LTE populations used as starting guess. In LTE mode,
  however, the structure of the file is different. The record length
  is still nz but now there are only two records. The first one
  contains n/g for the line lower level and then the same for the
  upper level. If multiple lines are being computed, only the level
  populations corresponding to the last line is written.
\item {\bf Output continuum opacity:} At the end of each synthesis,
  create a binary file named Cont\_opacity.dat. This file is
  overwritten after each synthesis, including those part of the
  inversion process. It doesn't work in parallel mode, as all
  processes would be writing concurrently in the same file. The file
  has binary format and has two records of length nz. The first record
  contains the continuum opacity at the wavelength of the last
  transition. The second record contains the reference continuum
  opacity at 500~nm.
\item {\bf Output NLTE source function:} At the end of each synthesis,
  create a binary file named NLTE\_sf.dat. This file is overwritten
  after each synthesis, including those part of the inversion
  process. It doesn't work in parallel mode, as all processes would be
  writing concurrently in the same file. The file has binary format
  and has record length nz. The records are arranged nested first in
  transitions (including bound-bound and bound-free transitions
  treated in detail), then an internal loop in frequencies from 1 to
  the number of wavelengths NQ specified in the model atom for that
  transition. Each record contains the source function used in the
  NLTE calculation for that transition and frequency.
\end{itemize}

Profiling is enabled in NICOLE by default. If for some reason you wish
to change this, edit the
profiling.f90 file in time\_code and change the following line near the
beginning:
\begin{verbatim}
  Logical, Save :: Do_profile=.True.
\end{verbatim}
to
\begin{verbatim}
  Logical, Save :: Do_profile=.False.
\end{verbatim}
It is not recommended to disable profiling because it doesn't have any
noticeable impact on performance and it produces valuable information
that can be used to optimize NICOLE and diagnose possible bottlenecks
in code execution. At the end of the run NICOLE produces an ASCII file 
named profiling\_\_0.txt (in the case of a parallel run there will be
one file for each process, each one with the process number n in the
file name, e.g. profiling\_\_4.txt). The contents of this file are
self-explanatory, with the total execution time as well as detailed 
information for each one of the major routines showing the number
of times it has been called, the time spent inside it and what
percentage of the total execution time it makes up for.

%%%%%%%%%%%%%%%%%%%%%%%%%%%%%%%%%%%%%%%%%%%%%%%%%%%%%%%%%%%%%%%%%%%%%%%
\chapter{Compiling NICOLE}

\section{Creating the makefile}
\label{sec:compiling}

 
NICOLE uses a few routines from the Numerical Recipes book (Press {\em
  et al.} 1986).  Unfortunately, due to licensing and copyright issues
I am not allowed to include the source code of these routines in the
distributions, so you will have to obtain them by yourself. You can
type the routines directly from the book, obtain the distribution on a
CD-ROM or download them from their website http://www.nr.com.

You must place the following routines in the numerical\_recipes/
directory: convlv.f90, four1.f90, fourrow.f90, nr.f90, nrtype.f90,
nrutil.f90, pythag.f90, realft.f90, svbksb.f90, svdcmp.f90 and
twofft.f90.

New in v2.0 NICOLE has a system analyzer written in Python that will
examine your system and automatically create a makefile, similarly to
the popular autoconf tool for Linux. Go to the main/ directory and try
the following:

\begin{verbatim}
./create_makefile.py
\end{verbatim}

This program will analyze your system and make some slight
modifications to the source files to adapt it to your architecture. If
everything goes well it will prompt you for authorization to create
a new makefile overwriting the old one. If you your compiler was not
automatically detected or if you wish to specify a different compiler,
use the -h command line flag:

\begin{verbatim}
./create_makefile.py -h
\end{verbatim}

If you wish, you can include optimization and debugging flags by using
--otherflags. For instance:
\begin{verbatim}
./create_makefile.py --otherflags='-g -O3'
\end{verbatim}

Once you have the makefile, simply type:
\begin{verbatim}
make clean
make nicole
\end{verbatim}


If everything goes well, you will end up with a fresh new 
executable nicole file in this directory. If you make any
modifications to the files with extension .presource, you will need to
run create\_makefile.py again before you can recompile with make nicole.

\section{Compiler notes} 

\subsection{Mac and GNU Fortran}
The Mac versions of GNU fortran prior to 4.4.x have a bug that make
NICOLE crash during the tests 2 to 6 (the first test runs
satisfactorily). Please, make sure that you are using a recent version
of gfortran for Mac (at least 4.4.x) to avoid this bug. Use gfortran
--version to find out your compiler version.

\subsection{Linux and GNU Fortran}
GNU fortran versions 4.4.x in Linux produce a NICOLE executable with
two problems: 1)the output is buffered so once you launch the code you
may not see anything at all until the excution finishes or the buffer
is full. Then all the output is flushed to the terminal at the same
time. 2) In inversion mode, the output model file will contain random
empty records (records filled with 0s). These problems do not appear
with versions 4.6.x.

\subsection{Intel Fortran}
The Intel fortran compiler ifort currently has a bug that makes it
place all array temporaries on the stack, regardless of their
size. An example may be found here:
\begin{verbatim}
http://stackoverflow.com/questions/12167549/
program-crash-for-array-copy-with-ifort
\end{verbatim}
This can make NICOLE crash under some circumstances. A simple
workaround is to use the -heap-arrays flag to specify the maximum array
size that may be placed on the stack, i.e.:
\begin{verbatim}
./create_makefile.py --otherflags='-fast -O3 -heap-arrays 1600'
\end{verbatim}


\section{MPI version}
\label{sec:MPI}

In order to compile the MPI version for parallel computers, use the
--mpi command line switch for create\_makefile:
\begin{verbatim}
./create_makefile.py --mpi
\end{verbatim}

You can also specify optimization flags with --otherflags as
before. Then, compile as usual:
\begin{verbatim}
make clean
make nicole
\end{verbatim}

\section{Testing the code}

Once you have successfully built the code, you can test it with a
battery of standard problems:
\begin{verbatim}
cd ../test
./run_tests.py
\end{verbatim}
This will launch the code on several different problems, make sure it
runs without crashes and analyze the results to ensure that they are
correct. 

For the parallel MPI build, you will have to use the --nicolecommand
switch to specify how the code is executed. The following is an
example:
\begin{verbatim}
cd ../test
./run_tests.py --nicolecommand='mpirun -n 8 ../../main/nicole'
\end{verbatim}


\subsection{Testing in non-interactive mode}

This simple script will not work in situations where the code is not
run interactively. This is typically the case when one runs in
parallel and/or using a queue system. In that case the procedure is
slightly more manual (but still manageable). 

Start by cleaning out any previously existing results in the test
subdirectories by running ./run\_tests.py --clean. Then
go into syn1/ and execute the Python launcher. Use the
--nicolecommand option to specify how to launch a program in that
system. For example, to run on two processors using mpich2 on my
dual-core laptop I would do: 
\begin{verbatim}
./run_nicole.py --clean
cd syn1/
./run_nicole.py --nicolecommand='mpirun -n 2 ../../main/nicole'
\end{verbatim}
If for some reason this procedure doesn't work, just do a dry run with
the Python launcher to prepare the input files and then run nicole
manually, like this:
\begin{verbatim}
./run_nicole.py --clean
cd syn1/
./run_nicole.py --nicolecommand='date'
mpirun -n 2 ../../main/nicole
\end{verbatim}

Repeat this in each one of the subdirectories under test/ (syn1,
syn2, ..., inv3). Then back to test/ and execute run\_test with the
--check-only switch. This switch is used to preserve the files in the
directory (created with your manual runs before) and check the results
in those files.
\begin{verbatim}
cd ..
./run_tests.py --check-only
\end{verbatim}

Just to show another example, this is how we would run the tests on
the LaPalma supercomputer of the Instituto de Astrof\'\i sica de
Canarias:
\begin{verbatim}
cd test/
./run_tests.py --clear
cd syn1/
./run_nicole.py --nicolecommand='mnsubmit ../jobscript.bash'
cd ../syn2
  (...repeat for each directory)
cd ../inv3
./run_nicole.py --nicolecommand='mnsubmit ../jobscript.bash'

cd ..
./run_tests.py --check-only
\end{verbatim}

\section{Compiling in double precision}
\label{sec:double}

There are plans to support a switch to select the desired precision
at compilation time but this is not yet possible in the current 
NICOLE version. However, there is a quick-and-dirty workaround using 
flags that most compilers incorporate to select the kind of implicit 
types. For example, the Intel compiler uses the flag -r8 to signal
the compiler that all implicit reals should be of kind=8 (i.e., double
precision). We can use this flag alongside a small modification to one
of the Numerical Recipes routines to compile NICOLE in double precision.
In the directory numerical\_recipes, edit the nrtype.f90 file and
change the following section:
\begin{verbatim}
  INTEGER, PARAMETER :: SP = KIND(1.0)
  INTEGER, PARAMETER :: DP = KIND(1.0D0)
  INTEGER, PARAMETER :: SPC = KIND((1.0,1.0))
  INTEGER, PARAMETER :: DPC = KIND((1.0D0,1.0D0))
\end{verbatim}
to:
\begin{verbatim}
  INTEGER, PARAMETER :: SP = 8
  INTEGER, PARAMETER :: DP = 4
  INTEGER, PARAMETER :: SPC = 8
  INTEGER, PARAMETER :: DPC = 4
\end{verbatim}

To compile with the -r8 (or whatever your compiler uses) flag, use the
--otherflags switch when creating the makefile (see 
Section~\ref{sec:compiling}). In our example:
\begin{verbatim}
./create_makefile.py --compiler=ifort --otherflags='-fast -r8'
\end{verbatim}
For gfortran, the equivalent to -r8 would be -fdefault-real-8 -fdefault-double-8 

\section{Supported platforms}

NICOLE has been tested on a number of different platforms, including
the following:

\begin{itemize}
\item PC Linux (i686), 32 bit with gfortran, ifort and pgf90 compilers
\item PC Linux (i686), 64 bit with gfortran, ifort and pgf90 compilers
\item Mac OS, Intel 32 bit with gfortran and ifort compilers
\item Mac OS, Intel 64 bit with gfortran and ifort compilers
\item PowerPC Linux 64 bit with xlf90 compiler
\end{itemize}

If you have had success compiling and running the code on other
platforms, please let me know the details (and whether you had to work
around any problems) by sending email to hsocas@iac.es

\chapter{The source code}

In the full distribution you will find the following subdirectories: 
\begin{itemize}
\item main: This is where you will find the main program nicole.f90, 
      some relevant subroutines and the makefile. The executable 
      file will be placed here when you compile NICOLE.
\item lorien: This contains the LORIEN kernel, consisting of the 
      lorien module (in the lorien.f90 source file) and all the 
      required subroutines. Check the LORIEN documentation
      included in its distribution (that you can obtain at the 
      C.I.C, homepage) for detailed information on these subroutines.
\item forward: You can find here the module needed for solving 
      the forward problem, i.e. given a model atmosphere, 
      synthesize the emergent profiles. The main routine is forward, 
      in the forward.f90 source file. 
\item compex: Here reside the compress and expand routines 
      (in the compress.f90 and expand.f90 source files), and their 
      associated subroutines. These routines are used to
      compress a model atmosphere onto a vector of dimensionless free 
      parameters, and vice versa, i.e.\ to expand this free parameters 
      vector to a model atmosphere.
\item time\_code: Routines to profile the code (determine where most
      of the time is spent)
\item misc: These are miscellaneous routines used to print out information, 
      read and write models, profiles, etc.
\item run: This is the directory where you can run the examples described 
      in this manual.
      You will find here the input files referenced in chapter~2
\item test: After compiling the code, try running the tests in this
  directory. Simply run the Python script run\_tests.py
\item numerical\_recipes: Place here the source files from the Numerical Recipes book
\item idl: Some useful IDL procedures to read/write files in various formats
\end{itemize}

\section{The dependency tree}
The following diagram illustrates the NICOLE dependency tree fully
expanded, showing all the dependencies required by a given file. This
can be obtained by running the following command in the main/
directory:
\begin{verbatim}
./create_makefile.py --showtree
\end{verbatim}

\begin{verbatim}
File:  ../compex/model_struct.f90
....../main/param_struct.f90
........../main/phys_constants.f90
File:  ../compex/select_number_of_nodes.f90
....../compex/nodes_info.f90
........../compex/model_struct.f90
............../main/param_struct.f90
................../main/phys_constants.f90
....../main/param_struct.f90
........../main/phys_constants.f90
....../compex/model_struct.f90
........../main/param_struct.f90
............../main/phys_constants.f90
File:  ../compex/compex.f90
....../main/param_struct.f90
........../main/phys_constants.f90
....../compex/model_struct.f90
........../main/param_struct.f90
............../main/phys_constants.f90
....../compex/nodes_info.f90
........../compex/model_struct.f90
............../main/param_struct.f90
................../main/phys_constants.f90
....../forward/bezier.f90
File:  ../compex/compress_variable.f90
File:  ../compex/randomize_model.f90
....../main/param_struct.f90
........../main/phys_constants.f90
....../compex/nodes_info.f90
........../compex/model_struct.f90
............../main/param_struct.f90
................../main/phys_constants.f90
....../compex/model_struct.f90
........../main/param_struct.f90
............../main/phys_constants.f90
File:  ../compex/record_to_model.f90
....../compex/model_struct.f90
........../main/param_struct.f90
............../main/phys_constants.f90
File:  ../compex/nodes_info.f90
....../compex/model_struct.f90
........../main/param_struct.f90
............../main/phys_constants.f90
File:  ../forward/atomic_data.f90
File:  ../forward/eq_state.f90
....../time_code/profiling.f90
....../numerical_recipes/nrtype.f90
....../main/param_struct.f90
........../main/phys_constants.f90
....../forward/atomic_data.f90
....../main/debug.f90
....../forward/lte.f90
........../main/param_struct.f90
............../main/phys_constants.f90
........../compex/model_struct.f90
............../main/param_struct.f90
................../main/phys_constants.f90
........../forward/line_data_struct.f90
........../forward/atomic_data.f90
File:  ../forward/lte.f90
....../main/param_struct.f90
........../main/phys_constants.f90
....../compex/model_struct.f90
........../main/param_struct.f90
............../main/phys_constants.f90
....../forward/line_data_struct.f90
....../forward/atomic_data.f90
File:  ../forward/line_data_struct.f90
File:  ../forward/zeeman_splitting.f90
File:  ../forward/forward.f90
....../forward/forward_supp.f90
........../main/phys_constants.f90
........../forward/atomic_data.f90
........../forward/line_data_struct.f90
........../forward/eq_state.f90
............../time_code/profiling.f90
............../numerical_recipes/nrtype.f90
............../main/param_struct.f90
................../main/phys_constants.f90
............../forward/atomic_data.f90
............../main/debug.f90
............../forward/lte.f90
................../main/param_struct.f90
....................../main/phys_constants.f90
................../compex/model_struct.f90
....................../main/param_struct.f90
........................../main/phys_constants.f90
................../forward/line_data_struct.f90
................../forward/atomic_data.f90
........../main/debug.f90
....../main/param_struct.f90
........../main/phys_constants.f90
....../compex/model_struct.f90
........../main/param_struct.f90
............../main/phys_constants.f90
....../forward/eq_state.f90
........../time_code/profiling.f90
........../numerical_recipes/nrtype.f90
........../main/param_struct.f90
............../main/phys_constants.f90
........../forward/atomic_data.f90
........../main/debug.f90
........../forward/lte.f90
............../main/param_struct.f90
................../main/phys_constants.f90
............../compex/model_struct.f90
................../main/param_struct.f90
....................../main/phys_constants.f90
............../forward/line_data_struct.f90
............../forward/atomic_data.f90
....../forward/zeeman_splitting.f90
....../forward/lte.f90
........../main/param_struct.f90
............../main/phys_constants.f90
........../compex/model_struct.f90
............../main/param_struct.f90
................../main/phys_constants.f90
........../forward/line_data_struct.f90
........../forward/atomic_data.f90
....../forward/NLTE/NLTE.f90
........../misc/file_ops.f90
........../main/param_struct.f90
............../main/phys_constants.f90
........../compex/model_struct.f90
............../main/param_struct.f90
................../main/phys_constants.f90
........../forward/line_data_struct.f90
........../forward/forward_supp.f90
............../main/phys_constants.f90
............../forward/atomic_data.f90
............../forward/line_data_struct.f90
............../forward/eq_state.f90
................../time_code/profiling.f90
................../numerical_recipes/nrtype.f90
................../main/param_struct.f90
....................../main/phys_constants.f90
................../forward/atomic_data.f90
................../main/debug.f90
................../forward/lte.f90
....................../main/param_struct.f90
........................../main/phys_constants.f90
....................../compex/model_struct.f90
........................../main/param_struct.f90
............................../main/phys_constants.f90
....................../forward/line_data_struct.f90
....................../forward/atomic_data.f90
............../main/debug.f90
........../forward/gauss_quad.f90
........../forward/lte.f90
............../main/param_struct.f90
................../main/phys_constants.f90
............../compex/model_struct.f90
................../main/param_struct.f90
....................../main/phys_constants.f90
............../forward/line_data_struct.f90
............../forward/atomic_data.f90
........../forward/eq_state.f90
............../time_code/profiling.f90
............../numerical_recipes/nrtype.f90
............../main/param_struct.f90
................../main/phys_constants.f90
............../forward/atomic_data.f90
............../main/debug.f90
............../forward/lte.f90
................../main/param_struct.f90
....................../main/phys_constants.f90
................../compex/model_struct.f90
....................../main/param_struct.f90
........................../main/phys_constants.f90
................../forward/line_data_struct.f90
................../forward/atomic_data.f90
........../forward/background.f90
............../forward/atomic_data.f90
............../main/debug.f90
........../forward/zeeman_splitting.f90
........../forward/bezier.f90
........../main/debug.f90
........../time_code/profiling.f90
........../main/phys_constants.f90
....../forward/atomic_data.f90
....../forward/gauss_quad.f90
....../forward/line_data_struct.f90
....../forward/bezier.f90
....../forward/background.f90
........../forward/atomic_data.f90
........../main/debug.f90
....../numerical_recipes/nr.f90
........../numerical_recipes/nrtype.f90
....../misc/file_ops.f90
....../time_code/profiling.f90
....../main/debug.f90
File:  ../forward/gauss_quad.f90
File:  ../forward/hydrostatic.f90
....../main/param_struct.f90
........../main/phys_constants.f90
....../compex/model_struct.f90
........../main/param_struct.f90
............../main/phys_constants.f90
....../forward/eq_state.f90
........../time_code/profiling.f90
........../numerical_recipes/nrtype.f90
........../main/param_struct.f90
............../main/phys_constants.f90
........../forward/atomic_data.f90
........../main/debug.f90
........../forward/lte.f90
............../main/param_struct.f90
................../main/phys_constants.f90
............../compex/model_struct.f90
................../main/param_struct.f90
....................../main/phys_constants.f90
............../forward/line_data_struct.f90
............../forward/atomic_data.f90
....../forward/atomic_data.f90
....../forward/background.f90
........../forward/atomic_data.f90
........../main/debug.f90
....../main/debug.f90
....../time_code/profiling.f90
File:  ../forward/background.f90
....../forward/atomic_data.f90
....../main/debug.f90
File:  ../forward/forward_supp.f90
....../main/phys_constants.f90
....../forward/atomic_data.f90
....../forward/line_data_struct.f90
....../forward/eq_state.f90
........../time_code/profiling.f90
........../numerical_recipes/nrtype.f90
........../main/param_struct.f90
............../main/phys_constants.f90
........../forward/atomic_data.f90
........../main/debug.f90
........../forward/lte.f90
............../main/param_struct.f90
................../main/phys_constants.f90
............../compex/model_struct.f90
................../main/param_struct.f90
....................../main/phys_constants.f90
............../forward/line_data_struct.f90
............../forward/atomic_data.f90
....../main/debug.f90
File:  ../forward/bezier.f90
File:  ../forward/ann/ann_pefrompg.f90
....../forward/ann/ann_pefrompg_data.f90
....../main/debug.f90
File:  ../forward/ann/ann_nh2frompe_data.f90
File:  ../forward/ann/ann_background_opacity.f90
....../forward/ann/ann_background_opacity_data.f90
....../time_code/profiling.f90
....../main/debug.f90
....../forward/background.f90
........../forward/atomic_data.f90
........../main/debug.f90
File:  ../forward/ann/ann_nh2frompe.f90
....../forward/ann/ann_nh2frompe_data.f90
....../main/debug.f90
File:  ../forward/ann/ann_nhfrompe.f90
....../forward/ann/ann_nhfrompe_data.f90
....../main/debug.f90
File:  ../forward/ann/ann_pgfrompe.f90
....../forward/ann/ann_pgfrompe_data.f90
....../main/debug.f90
File:  ../forward/ann/ann_pgfrompe_data.f90
File:  ../forward/ann/ann_nhfrompe_data.f90
File:  ../forward/ann/ann_background_opacity_data.f90
File:  ../forward/ann/ann_pefrompg_data.f90
File:  ../forward/ann/ANN_forward.f90
File:  ../forward/hyperfine/zeeman_hyperfine.f90
....../forward/line_data_struct.f90
....../forward/hyperfine/hyperfine.f90
........../numerical_recipes/nr.f90
............../numerical_recipes/nrtype.f90
........../main/param_struct.f90
............../main/phys_constants.f90
....../main/param_struct.f90
........../main/phys_constants.f90
....../time_code/profiling.f90
File:  ../forward/hyperfine/hyperfine.f90
....../numerical_recipes/nr.f90
........../numerical_recipes/nrtype.f90
....../main/param_struct.f90
........../main/phys_constants.f90
File:  ../forward/sopa/sopa.f90
File:  ../forward/NLTE/NLTE.f90
....../misc/file_ops.f90
....../main/param_struct.f90
........../main/phys_constants.f90
....../compex/model_struct.f90
........../main/param_struct.f90
............../main/phys_constants.f90
....../forward/line_data_struct.f90
....../forward/forward_supp.f90
........../main/phys_constants.f90
........../forward/atomic_data.f90
........../forward/line_data_struct.f90
........../forward/eq_state.f90
............../time_code/profiling.f90
............../numerical_recipes/nrtype.f90
............../main/param_struct.f90
................../main/phys_constants.f90
............../forward/atomic_data.f90
............../main/debug.f90
............../forward/lte.f90
................../main/param_struct.f90
....................../main/phys_constants.f90
................../compex/model_struct.f90
....................../main/param_struct.f90
........................../main/phys_constants.f90
................../forward/line_data_struct.f90
................../forward/atomic_data.f90
........../main/debug.f90
....../forward/gauss_quad.f90
....../forward/lte.f90
........../main/param_struct.f90
............../main/phys_constants.f90
........../compex/model_struct.f90
............../main/param_struct.f90
................../main/phys_constants.f90
........../forward/line_data_struct.f90
........../forward/atomic_data.f90
....../forward/eq_state.f90
........../time_code/profiling.f90
........../numerical_recipes/nrtype.f90
........../main/param_struct.f90
............../main/phys_constants.f90
........../forward/atomic_data.f90
........../main/debug.f90
........../forward/lte.f90
............../main/param_struct.f90
................../main/phys_constants.f90
............../compex/model_struct.f90
................../main/param_struct.f90
....................../main/phys_constants.f90
............../forward/line_data_struct.f90
............../forward/atomic_data.f90
....../forward/background.f90
........../forward/atomic_data.f90
........../main/debug.f90
....../forward/zeeman_splitting.f90
....../forward/bezier.f90
....../main/debug.f90
....../time_code/profiling.f90
....../main/phys_constants.f90
File:  ../main/phys_constants.f90
File:  ../main/param_struct.f90
....../main/phys_constants.f90
File:  ../main/nicole.f90
....../main/param_struct.f90
........../main/phys_constants.f90
....../forward/line_data_struct.f90
....../misc/nicole_input.f90
....../compex/compex.f90
........../main/param_struct.f90
............../main/phys_constants.f90
........../compex/model_struct.f90
............../main/param_struct.f90
................../main/phys_constants.f90
........../compex/nodes_info.f90
............../compex/model_struct.f90
................../main/param_struct.f90
....................../main/phys_constants.f90
........../forward/bezier.f90
....../compex/model_struct.f90
........../main/param_struct.f90
............../main/phys_constants.f90
....../lorien/lorien.f90
........../main/param_struct.f90
............../main/phys_constants.f90
........../compex/compex.f90
............../main/param_struct.f90
................../main/phys_constants.f90
............../compex/model_struct.f90
................../main/param_struct.f90
....................../main/phys_constants.f90
............../compex/nodes_info.f90
................../compex/model_struct.f90
....................../main/param_struct.f90
........................../main/phys_constants.f90
............../forward/bezier.f90
........../forward/forward.f90
............../forward/forward_supp.f90
................../main/phys_constants.f90
................../forward/atomic_data.f90
................../forward/line_data_struct.f90
................../forward/eq_state.f90
....................../time_code/profiling.f90
....................../numerical_recipes/nrtype.f90
....................../main/param_struct.f90
........................../main/phys_constants.f90
....................../forward/atomic_data.f90
....................../main/debug.f90
....................../forward/lte.f90
........................../main/param_struct.f90
............................../main/phys_constants.f90
........................../compex/model_struct.f90
............................../main/param_struct.f90
................................../main/phys_constants.f90
........................../forward/line_data_struct.f90
........................../forward/atomic_data.f90
................../main/debug.f90
............../main/param_struct.f90
................../main/phys_constants.f90
............../compex/model_struct.f90
................../main/param_struct.f90
....................../main/phys_constants.f90
............../forward/eq_state.f90
................../time_code/profiling.f90
................../numerical_recipes/nrtype.f90
................../main/param_struct.f90
....................../main/phys_constants.f90
................../forward/atomic_data.f90
................../main/debug.f90
................../forward/lte.f90
....................../main/param_struct.f90
........................../main/phys_constants.f90
....................../compex/model_struct.f90
........................../main/param_struct.f90
............................../main/phys_constants.f90
....................../forward/line_data_struct.f90
....................../forward/atomic_data.f90
............../forward/zeeman_splitting.f90
............../forward/lte.f90
................../main/param_struct.f90
....................../main/phys_constants.f90
................../compex/model_struct.f90
....................../main/param_struct.f90
........................../main/phys_constants.f90
................../forward/line_data_struct.f90
................../forward/atomic_data.f90
............../forward/NLTE/NLTE.f90
................../misc/file_ops.f90
................../main/param_struct.f90
....................../main/phys_constants.f90
................../compex/model_struct.f90
....................../main/param_struct.f90
........................../main/phys_constants.f90
................../forward/line_data_struct.f90
................../forward/forward_supp.f90
....................../main/phys_constants.f90
....................../forward/atomic_data.f90
....................../forward/line_data_struct.f90
....................../forward/eq_state.f90
........................../time_code/profiling.f90
........................../numerical_recipes/nrtype.f90
........................../main/param_struct.f90
............................../main/phys_constants.f90
........................../forward/atomic_data.f90
........................../main/debug.f90
........................../forward/lte.f90
............................../main/param_struct.f90
................................../main/phys_constants.f90
............................../compex/model_struct.f90
................................../main/param_struct.f90
....................................../main/phys_constants.f90
............................../forward/line_data_struct.f90
............................../forward/atomic_data.f90
....................../main/debug.f90
................../forward/gauss_quad.f90
................../forward/lte.f90
....................../main/param_struct.f90
........................../main/phys_constants.f90
....................../compex/model_struct.f90
........................../main/param_struct.f90
............................../main/phys_constants.f90
....................../forward/line_data_struct.f90
....................../forward/atomic_data.f90
................../forward/eq_state.f90
....................../time_code/profiling.f90
....................../numerical_recipes/nrtype.f90
....................../main/param_struct.f90
........................../main/phys_constants.f90
....................../forward/atomic_data.f90
....................../main/debug.f90
....................../forward/lte.f90
........................../main/param_struct.f90
............................../main/phys_constants.f90
........................../compex/model_struct.f90
............................../main/param_struct.f90
................................../main/phys_constants.f90
........................../forward/line_data_struct.f90
........................../forward/atomic_data.f90
................../forward/background.f90
....................../forward/atomic_data.f90
....................../main/debug.f90
................../forward/zeeman_splitting.f90
................../forward/bezier.f90
................../main/debug.f90
................../time_code/profiling.f90
................../main/phys_constants.f90
............../forward/atomic_data.f90
............../forward/gauss_quad.f90
............../forward/line_data_struct.f90
............../forward/bezier.f90
............../forward/background.f90
................../forward/atomic_data.f90
................../main/debug.f90
............../numerical_recipes/nr.f90
................../numerical_recipes/nrtype.f90
............../misc/file_ops.f90
............../time_code/profiling.f90
............../main/debug.f90
........../compex/model_struct.f90
............../main/param_struct.f90
................../main/phys_constants.f90
........../forward/line_data_struct.f90
........../compex/nodes_info.f90
............../compex/model_struct.f90
................../main/param_struct.f90
....................../main/phys_constants.f90
........../time_code/profiling.f90
........../main/debug.f90
....../forward/forward.f90
........../forward/forward_supp.f90
............../main/phys_constants.f90
............../forward/atomic_data.f90
............../forward/line_data_struct.f90
............../forward/eq_state.f90
................../time_code/profiling.f90
................../numerical_recipes/nrtype.f90
................../main/param_struct.f90
....................../main/phys_constants.f90
................../forward/atomic_data.f90
................../main/debug.f90
................../forward/lte.f90
....................../main/param_struct.f90
........................../main/phys_constants.f90
....................../compex/model_struct.f90
........................../main/param_struct.f90
............................../main/phys_constants.f90
....................../forward/line_data_struct.f90
....................../forward/atomic_data.f90
............../main/debug.f90
........../main/param_struct.f90
............../main/phys_constants.f90
........../compex/model_struct.f90
............../main/param_struct.f90
................../main/phys_constants.f90
........../forward/eq_state.f90
............../time_code/profiling.f90
............../numerical_recipes/nrtype.f90
............../main/param_struct.f90
................../main/phys_constants.f90
............../forward/atomic_data.f90
............../main/debug.f90
............../forward/lte.f90
................../main/param_struct.f90
....................../main/phys_constants.f90
................../compex/model_struct.f90
....................../main/param_struct.f90
........................../main/phys_constants.f90
................../forward/line_data_struct.f90
................../forward/atomic_data.f90
........../forward/zeeman_splitting.f90
........../forward/lte.f90
............../main/param_struct.f90
................../main/phys_constants.f90
............../compex/model_struct.f90
................../main/param_struct.f90
....................../main/phys_constants.f90
............../forward/line_data_struct.f90
............../forward/atomic_data.f90
........../forward/NLTE/NLTE.f90
............../misc/file_ops.f90
............../main/param_struct.f90
................../main/phys_constants.f90
............../compex/model_struct.f90
................../main/param_struct.f90
....................../main/phys_constants.f90
............../forward/line_data_struct.f90
............../forward/forward_supp.f90
................../main/phys_constants.f90
................../forward/atomic_data.f90
................../forward/line_data_struct.f90
................../forward/eq_state.f90
....................../time_code/profiling.f90
....................../numerical_recipes/nrtype.f90
....................../main/param_struct.f90
........................../main/phys_constants.f90
....................../forward/atomic_data.f90
....................../main/debug.f90
....................../forward/lte.f90
........................../main/param_struct.f90
............................../main/phys_constants.f90
........................../compex/model_struct.f90
............................../main/param_struct.f90
................................../main/phys_constants.f90
........................../forward/line_data_struct.f90
........................../forward/atomic_data.f90
................../main/debug.f90
............../forward/gauss_quad.f90
............../forward/lte.f90
................../main/param_struct.f90
....................../main/phys_constants.f90
................../compex/model_struct.f90
....................../main/param_struct.f90
........................../main/phys_constants.f90
................../forward/line_data_struct.f90
................../forward/atomic_data.f90
............../forward/eq_state.f90
................../time_code/profiling.f90
................../numerical_recipes/nrtype.f90
................../main/param_struct.f90
....................../main/phys_constants.f90
................../forward/atomic_data.f90
................../main/debug.f90
................../forward/lte.f90
....................../main/param_struct.f90
........................../main/phys_constants.f90
....................../compex/model_struct.f90
........................../main/param_struct.f90
............................../main/phys_constants.f90
....................../forward/line_data_struct.f90
....................../forward/atomic_data.f90
............../forward/background.f90
................../forward/atomic_data.f90
................../main/debug.f90
............../forward/zeeman_splitting.f90
............../forward/bezier.f90
............../main/debug.f90
............../time_code/profiling.f90
............../main/phys_constants.f90
........../forward/atomic_data.f90
........../forward/gauss_quad.f90
........../forward/line_data_struct.f90
........../forward/bezier.f90
........../forward/background.f90
............../forward/atomic_data.f90
............../main/debug.f90
........../numerical_recipes/nr.f90
............../numerical_recipes/nrtype.f90
........../misc/file_ops.f90
........../time_code/profiling.f90
........../main/debug.f90
....../compex/nodes_info.f90
........../compex/model_struct.f90
............../main/param_struct.f90
................../main/phys_constants.f90
....../forward/atomic_data.f90
....../misc/file_ops.f90
....../main/debug.f90
....../time_code/profiling.f90
....../forward/background.f90
........../forward/atomic_data.f90
........../main/debug.f90
File:  ../main/debug.f90
File:  ../time_code/profiling.f90
File:  ../misc/checknan.f90
File:  ../misc/tolower.f90
File:  ../misc/parab_interp.f90
File:  ../misc/get_lun.f90
File:  ../misc/myflush.f90
File:  ../misc/roman_to_int.f90
File:  ../misc/set_params.f90
....../misc/nicole_input.f90
....../main/param_struct.f90
........../main/phys_constants.f90
File:  ../misc/tau_to_z.f90
....../main/param_struct.f90
........../main/phys_constants.f90
....../compex/model_struct.f90
........../main/param_struct.f90
............../main/phys_constants.f90
....../forward/background.f90
........../forward/atomic_data.f90
........../main/debug.f90
File:  ../misc/write_profile.f90
....../main/param_struct.f90
........../main/phys_constants.f90
....../forward/line_data_struct.f90
File:  ../misc/read_weights.f90
....../main/param_struct.f90
........../main/phys_constants.f90
File:  ../misc/write_direct.f90
File:  ../misc/z_to_tau.f90
....../main/param_struct.f90
........../main/phys_constants.f90
....../forward/eq_state.f90
........../time_code/profiling.f90
........../numerical_recipes/nrtype.f90
........../main/param_struct.f90
............../main/phys_constants.f90
........../forward/atomic_data.f90
........../main/debug.f90
........../forward/lte.f90
............../main/param_struct.f90
................../main/phys_constants.f90
............../compex/model_struct.f90
................../main/param_struct.f90
....................../main/phys_constants.f90
............../forward/line_data_struct.f90
............../forward/atomic_data.f90
....../compex/model_struct.f90
........../main/param_struct.f90
............../main/phys_constants.f90
....../forward/atomic_data.f90
....../forward/background.f90
........../forward/atomic_data.f90
........../main/debug.f90
....../main/debug.f90
File:  ../misc/file_ops.f90
File:  ../misc/printout.f90
....../main/param_struct.f90
........../main/phys_constants.f90
....../compex/model_struct.f90
........../main/param_struct.f90
............../main/phys_constants.f90
....../forward/line_data_struct.f90
File:  ../misc/cubic_interp.f90
File:  ../misc/los_to_local.f90
File:  ../misc/smoothed_lines.f90
File:  ../misc/local_to_los.f90
File:  ../misc/read_next_nocomment.f90
File:  ../misc/nicole_input.f90
File:  ../misc/read_profile.f90
....../main/param_struct.f90
........../main/phys_constants.f90
File:  ../misc/compute_weights.f90
....../main/param_struct.f90
........../main/phys_constants.f90
File:  ../misc/los_to_local_errors.f90
File:  ../misc/write_model.f90
....../main/param_struct.f90
........../main/phys_constants.f90
....../compex/model_struct.f90
........../main/param_struct.f90
............../main/phys_constants.f90
....../misc/file_ops.f90
....../forward/line_data_struct.f90
File:  ../misc/toupper.f90
File:  ../numerical_recipes/twofft.f90
....../numerical_recipes/nrtype.f90
....../numerical_recipes/nrutil.f90
........../numerical_recipes/nrtype.f90
....../numerical_recipes/nr.f90
........../numerical_recipes/nrtype.f90
File:  ../numerical_recipes/added.f90
....../numerical_recipes/nrtype.f90
File:  ../numerical_recipes/lubksb.f90
....../numerical_recipes/nrtype.f90
....../numerical_recipes/nrutil.f90
........../numerical_recipes/nrtype.f90
File:  ../numerical_recipes/lubksb_dp.f90
....../numerical_recipes/nrtype.f90
....../numerical_recipes/nrutil.f90
........../numerical_recipes/nrtype.f90
File:  ../numerical_recipes/svbksb.f90
....../numerical_recipes/nrtype.f90
....../numerical_recipes/nrutil.f90
........../numerical_recipes/nrtype.f90
File:  ../numerical_recipes/realft.f90
....../numerical_recipes/nrtype.f90
....../numerical_recipes/nrutil.f90
........../numerical_recipes/nrtype.f90
....../numerical_recipes/nr.f90
........../numerical_recipes/nrtype.f90
File:  ../numerical_recipes/svdcmp.f90
....../numerical_recipes/nrtype.f90
....../numerical_recipes/nrutil.f90
........../numerical_recipes/nrtype.f90
....../main/debug.f90
....../numerical_recipes/nr.f90
........../numerical_recipes/nrtype.f90
File:  ../numerical_recipes/nr.f90
....../numerical_recipes/nrtype.f90
File:  ../numerical_recipes/pythag.f90
....../numerical_recipes/nrtype.f90
File:  ../numerical_recipes/ludcmp_dp.f90
....../numerical_recipes/nrtype.f90
....../main/debug.f90
....../numerical_recipes/added.f90
........../numerical_recipes/nrtype.f90
....../numerical_recipes/nrutil.f90
........../numerical_recipes/nrtype.f90
File:  ../numerical_recipes/four1.f90
....../numerical_recipes/nrtype.f90
....../numerical_recipes/nrutil.f90
........../numerical_recipes/nrtype.f90
....../numerical_recipes/nr.f90
........../numerical_recipes/nrtype.f90
File:  ../numerical_recipes/convlv.f90
....../numerical_recipes/nrtype.f90
....../numerical_recipes/nrutil.f90
........../numerical_recipes/nrtype.f90
....../numerical_recipes/nr.f90
........../numerical_recipes/nrtype.f90
File:  ../numerical_recipes/fourrow.f90
....../numerical_recipes/nrtype.f90
....../numerical_recipes/nrutil.f90
........../numerical_recipes/nrtype.f90
File:  ../numerical_recipes/tqli.f90
....../numerical_recipes/nrtype.f90
....../numerical_recipes/nrutil.f90
........../numerical_recipes/nrtype.f90
....../numerical_recipes/nr.f90
........../numerical_recipes/nrtype.f90
File:  ../numerical_recipes/nrtype.f90
File:  ../numerical_recipes/nrutil.f90
....../numerical_recipes/nrtype.f90
File:  ../numerical_recipes/ludcmp.f90
....../numerical_recipes/nrtype.f90
....../numerical_recipes/nrutil.f90
........../numerical_recipes/nrtype.f90
File:  ../lorien/SVD_solve.f90
....../numerical_recipes/nr.f90
........../numerical_recipes/nrtype.f90
....../main/debug.f90
File:  ../lorien/lorien.f90
....../main/param_struct.f90
........../main/phys_constants.f90
....../compex/compex.f90
........../main/param_struct.f90
............../main/phys_constants.f90
........../compex/model_struct.f90
............../main/param_struct.f90
................../main/phys_constants.f90
........../compex/nodes_info.f90
............../compex/model_struct.f90
................../main/param_struct.f90
....................../main/phys_constants.f90
........../forward/bezier.f90
....../forward/forward.f90
........../forward/forward_supp.f90
............../main/phys_constants.f90
............../forward/atomic_data.f90
............../forward/line_data_struct.f90
............../forward/eq_state.f90
................../time_code/profiling.f90
................../numerical_recipes/nrtype.f90
................../main/param_struct.f90
....................../main/phys_constants.f90
................../forward/atomic_data.f90
................../main/debug.f90
................../forward/lte.f90
....................../main/param_struct.f90
........................../main/phys_constants.f90
....................../compex/model_struct.f90
........................../main/param_struct.f90
............................../main/phys_constants.f90
....................../forward/line_data_struct.f90
....................../forward/atomic_data.f90
............../main/debug.f90
........../main/param_struct.f90
............../main/phys_constants.f90
........../compex/model_struct.f90
............../main/param_struct.f90
................../main/phys_constants.f90
........../forward/eq_state.f90
............../time_code/profiling.f90
............../numerical_recipes/nrtype.f90
............../main/param_struct.f90
................../main/phys_constants.f90
............../forward/atomic_data.f90
............../main/debug.f90
............../forward/lte.f90
................../main/param_struct.f90
....................../main/phys_constants.f90
................../compex/model_struct.f90
....................../main/param_struct.f90
........................../main/phys_constants.f90
................../forward/line_data_struct.f90
................../forward/atomic_data.f90
........../forward/zeeman_splitting.f90
........../forward/lte.f90
............../main/param_struct.f90
................../main/phys_constants.f90
............../compex/model_struct.f90
................../main/param_struct.f90
....................../main/phys_constants.f90
............../forward/line_data_struct.f90
............../forward/atomic_data.f90
........../forward/NLTE/NLTE.f90
............../misc/file_ops.f90
............../main/param_struct.f90
................../main/phys_constants.f90
............../compex/model_struct.f90
................../main/param_struct.f90
....................../main/phys_constants.f90
............../forward/line_data_struct.f90
............../forward/forward_supp.f90
................../main/phys_constants.f90
................../forward/atomic_data.f90
................../forward/line_data_struct.f90
................../forward/eq_state.f90
....................../time_code/profiling.f90
....................../numerical_recipes/nrtype.f90
....................../main/param_struct.f90
........................../main/phys_constants.f90
....................../forward/atomic_data.f90
....................../main/debug.f90
....................../forward/lte.f90
........................../main/param_struct.f90
............................../main/phys_constants.f90
........................../compex/model_struct.f90
............................../main/param_struct.f90
................................../main/phys_constants.f90
........................../forward/line_data_struct.f90
........................../forward/atomic_data.f90
................../main/debug.f90
............../forward/gauss_quad.f90
............../forward/lte.f90
................../main/param_struct.f90
....................../main/phys_constants.f90
................../compex/model_struct.f90
....................../main/param_struct.f90
........................../main/phys_constants.f90
................../forward/line_data_struct.f90
................../forward/atomic_data.f90
............../forward/eq_state.f90
................../time_code/profiling.f90
................../numerical_recipes/nrtype.f90
................../main/param_struct.f90
....................../main/phys_constants.f90
................../forward/atomic_data.f90
................../main/debug.f90
................../forward/lte.f90
....................../main/param_struct.f90
........................../main/phys_constants.f90
....................../compex/model_struct.f90
........................../main/param_struct.f90
............................../main/phys_constants.f90
....................../forward/line_data_struct.f90
....................../forward/atomic_data.f90
............../forward/background.f90
................../forward/atomic_data.f90
................../main/debug.f90
............../forward/zeeman_splitting.f90
............../forward/bezier.f90
............../main/debug.f90
............../time_code/profiling.f90
............../main/phys_constants.f90
........../forward/atomic_data.f90
........../forward/gauss_quad.f90
........../forward/line_data_struct.f90
........../forward/bezier.f90
........../forward/background.f90
............../forward/atomic_data.f90
............../main/debug.f90
........../numerical_recipes/nr.f90
............../numerical_recipes/nrtype.f90
........../misc/file_ops.f90
........../time_code/profiling.f90
........../main/debug.f90
....../compex/model_struct.f90
........../main/param_struct.f90
............../main/phys_constants.f90
....../forward/line_data_struct.f90
....../compex/nodes_info.f90
........../compex/model_struct.f90
............../main/param_struct.f90
................../main/phys_constants.f90
....../time_code/profiling.f90
....../main/debug.f90
File:  ../lorien/adjust_wave_grid.f90
....../main/param_struct.f90
........../main/phys_constants.f90
\end{verbatim}

\chapter{Geometry of the magnetic field}

The inclination and azimuth given in the atmospheric models as
B\_local\_inc and B\_local\_azi refer to the local solar frame.  This
means that the inclination is given with respect to the solar vertical
and the azimuth is given with respect to the local E-W solar
direction (measured counterclock-wise from the West). When you specify
a heliocentric observation angle, NICOLE will use appropriate
coordinates conversions to calculate the apparent inclination and
azimuth (the inclination and azimuth with respect to the line of
sight) so that the emergent profiles obtained are those that one would
actually observe.

Unfortunately, a fully detailed calculation would be very complex, and
would require the user to introduce information such as the solar
pitch angle, which is not always available.  Therefore, for a sake of
simplicity and convenience, I have decided to assume that the observed
spot lies on the E-W line that passes through the center of the solar
disk. In this manner, only the heliocentric angle needs to be
specified and the coordinates conversion equations simplify to:

\begin{eqnarray}
\cos \gamma = \sin \gamma' \cos \chi' sin \theta + \cos\gamma' cos \theta, \\
\chi = \chi'
\end{eqnarray}

where $\gamma$ and $\chi$ are the local inclination and azimuth, 
$\gamma'$  and $\chi'$ are the line-of-sight inclination
and azimuth, and $\theta$ is the heliocentric angle of the 
observations. Simple expressions are also obtained for the 
conversion of the errors in these quantities.

\begin{equation}
\delta (\cos \gamma) = -\sin\gamma\delta\gamma =
\cos \gamma' \cos\chi' \sin\theta\delta\gamma'- 
\sin\gamma' \sin\chi' \sin\theta\delta\chi'- 
\sin\gamma' \cos \theta \delta\gamma'~.
\end{equation}

The above-mentioned approximation has no effect on the inferred magnetic 
inclination, but the magnetic azimuth resulting from an inversion will be 
correct only if we are observing a location on the solar disk equator. 
If this is not the case, there will be a constant offset in the
azimuth that can be corrected {\it a posteriori}. 
As a first-order correction, you may want to add
the heliocentric azimuth (i.e., the azimuth of the observed spot 
measured counterclock-wise from the West direction of the disk equator) 
to the inferred azimuth.


\chapter{Troubleshooting}
Below are some problems that you might run into from time to time:

\begin{itemize}

\item {\bf I have problem compiling file with MPI.} The current
  version of the code does not work with OpenMPI. If OpenMPI is the
  only MPI version available on your machine, you have to install
  mpich or Intel-MPI locally in your home directory.  to be sure that
  create$\_$makefile.py is using the correct version of the code, do the
  following: 
Execute
\begin{verbatim}
which mpif90
\end{verbatim}
If the result is not the correct MPI version, to fix that set the path to it in your .bashrc file and execute
\begin{verbatim}
source .bashrc.  
\end{verbatim}
Create make file with:
 \begin{verbatim}
./create_makefile --compiler='/absolutepathtoyourmpi/bin/mpif90
 \end{verbatim}
Be sure that the --mpi flag is NOT specified in this case.
\item {\bf I get an error about an incorrect magic number in a python
  script.} Remove all the .pyc files in your directory. These are
  compiled binaries that Python creates to speed things up. Perhaps
  the .pyc file it tried to read was created by a different machine.
\item {\bf The code crashes on startup.} This is probably caused by an
  incorrect format of one of the input files. Try replacing your input
  files, one by one, with the examples included in your
  distribution. This way you can track down which file is causing the
  problem.  Then, edit the file and check with the corresponding
  section in this manual to make sure that the format is exactly the
  one expected by NICOLE.
\item {\bf The NLTE iteration doesn't converge.} Try playing around
  with the parameters in the NLTE section of NICOLE.input. The first
  thing to try would be to use linear formal solution. If that doesn't
  work, try changing the formal solution.
\item {\bf My synthesis doesn't produce the results I expect.} Perhaps
  NICOLE is modifying some of the model parameters that you are
  supplying in the model (pressure, density, Hydrogen level
  populations, etc). Check the discussion at the end of section~\ref{sec:runsyn}. 
\item {\bf I can't get a good fit to my profiles.} Check the tips on
  section~\ref{sec:tips}. If that doesn't help, make sure that your
  profiles are properly normalized.
\item {\bf Something's weird with my inversions} Do you have a leftover
  Restart=Yes field in your NICOLE.input? 
\item {\bf I get ``line not optically thin at the surface'' errors
  during my inversions.}  This error means that your model does not
  extend high enough.  You should add more layers on top of it. For
  Fe~I~6301 and 6302 you should go up to about
  log($\tau_{5000}$)=$-4.5$.
\item {\bf I get ``line not optically thick at the bottom'' errors
  during my inversions.}  This error means that your model does not
  extend low enough. You should add more layers underneath it.
  Normally you should be fine if you go below log($\tau_{5000}$)=2.
\item {\bf I get ``hydrostatic inaccuracy'' errors during my
  inversions.}  This usually happens when temperatures go very low (or
  there is a steep temperature drop) in the upper layers.  Your lines
  are normally quite insensitive to these upper layers and that is why
  they do weird things from time to time.  But if you cut them out,
  you may start getting the ``not optically thin'' errors mentioned
  above. My advice is that you start iterating with this model, even
  if the lines are not getting optically thin at the surface. This
  means that your final result will not be accurate, but it will
  probably be close enough. Now add the extra layers on top of your
  model and invert again. That should get rid of the hydrostatic
  inaccuracy problem.
\item {\bf I made a synthesis using a quiet Sun simulation cube from
  my favorite MHD code and the average continuum is not at 1}. NICOLE
  uses a set of opacities that are, in general, different from those
  in the MHD code used to produce the simulation. This means that the
  $\tau=1$ level is not the same (assuming that you're using a
  geometrical height scale) and therefore one is seeing a slightly
  different layer. This is not really an error, it's just the result
  of combining results from two different codes and we need a different
  interpretation. Instead of relying on the NICOLE
  normalization it would be better to compute intensities from a quiet
  Sun snapshot and use its average value as the normalization reference.
\item {\bf I modified the source code and now something is going
  wrong.}  Try using the make clean command before compiling.
\item {\bf I have read through this manual and the troubleshooting
  section, and I still can't find a solution for my problem.} Ok, if
  you can't figure it out send an e-mail at hsocas@iac.es.
\end{itemize}

\chapter{Version history}

\begin{itemize}
\item v0.9 Original version, based on LILIA v4.1
\item v1.1  Departure coefficients, instrumental profile, multiple
  synthesis mode
\end{itemize}

... and at that point I stopped keeping track. Using GIT since v2.70
for version control

After v2.72, the version numbering scheme has changed to the format
YY.MM which represents the release year and month. The first such
release was 13.12 in December 2013.

\chapter{Bibliography}

\begin{itemize}
\item Asensio Ramos, A., Trujillo Bueno, J., Carlsson, M., Cernicharo,
  J. ApJ 2003, 588, L61
\item Bellot Rubio, L.R., Ruiz Cabo, B., and Collados Vera, M. A\&A 1998, 
      506, 805.
\item Barkelm, P. S., Piskunov, N., and O'Mara, B. J. A\/AS 2000, 142, 467.
\item Kostik, R. I., Shchukina, N. G., and Rutten, R. J. 1996, 
      A\&A, 305, 325.
\item Mihalas, D. in Methods in Computational Physics, Vol. 7, Alder,
  B. (ed.), New York; London; Academic Press 1967
\item Press, W. H., Flannery, B. P., Teukolsky, S. A., and Vetterling, 
      W. T. 1986, Numerical Recipes (Cambridge: Cambridge U niv. Press)
\item Ruiz Cabo, B., and del Taro Iniesta, J. C. 1992, ApJ, 398, 375.
\item Rybicki, G. B., \& Hummer, D. G. 1991, A\&A, 245, 171.
\item S\'anchez Almeida, J., and Trujillo Bueno, J. 1999, ApJ, 526,
  1013.
\item Scharmer, G. B., and Carlsson, M. 1985, J. Comput. Phys., 50,
  56.
\item Socas-Navarro, H., de la Cruz Rodr\'\i guez, J., Asensio Ramos, A., Trujillo Bueno, J. and Ruiz Cobo, B. 2014, A\&A, {\em in preparation}.
\item Socas-Navarro, H. and Trujillo Bueno, J. 1998, ApJ, 490, 383.
\item Unsold 1955, Physik der Sternatmospharen, 2nd ed
\end{itemize}


\end{document}
